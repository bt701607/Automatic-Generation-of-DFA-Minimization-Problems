% !TeX spellcheck = en_US

\documentclass[a4paper, oneside, 11pt]{report}

% ----------- packages -------------

\usepackage[a4paper,width=150mm,top=25mm,bottom=25mm]{geometry}

\usepackage[utf8]{inputenc}
\usepackage[english]{babel}
\usepackage[T1]{fontenc}
\usepackage{lmodern}

\usepackage{graphicx}
\graphicspath{{images/}}

\usepackage{amsmath}
\usepackage{amsthm}
\usepackage{amssymb}
\usepackage{wasysym}

\usepackage{algorithm}
\usepackage[noend]{algpseudocode}

\usepackage{setspace}
\usepackage{float}

\PassOptionsToPackage{hyphens}{url}\usepackage{hyperref}

\usepackage[dvipsnames]{xcolor}

\usepackage[nottoc,numbib]{tocbibind}
\bibliographystyle{plainurl}

% ----------- macros -------------

\newtheorem{theorem}{Theorem}
\newtheorem{corollary}{Corollary}
\newtheorem{lemma}{Lemma}
\newtheorem{proposition}{Proposition}

\theoremstyle{definition}
\newtheorem{definition}{Definition}

\theoremstyle{remark}
\newtheorem*{remark}{Remark}

\algdef{SE}[DOWHILE]{Do}{doWhile}{\algorithmicdo}[1]{\algorithmicwhile\ #1}%

\newcommand{\MinAlg}{\textsc{MinimizeDFA}}
\newcommand{\CompUnr}{\textsc{ComUnreachables}}
\newcommand{\RemUnr}{\textsc{RemUnreachables}}
\newcommand{\CompDist}{\textsc{ComEquivPairs}}
\newcommand{\RemEq}{\textsc{RemEquivPairs}}
%\newcommand{\mmD}{\mathcal{D}}
\newcommand{\mmD}{D}

\newcommand{\gregorColor}{Violet}
\newcommand{\gregor}[1]{\textcolor{\gregorColor}{\textbf{Gregor:} #1}}

% ----------- content -------------

\title{Generation of DFA Minimization Problems}
\author{Gregor Hans Christian Sönnichsen}

\begin{document}
	
	\maketitle
	
	\begin{abstract}
        The theory of deterministic finite automatons (DFAs) is a classical topic of computer science-related courses. A typical task for students is to minimize a DFA by deleting irrelevant elements. However generation of those DFAs that shall be minimized is often done manually by the exercise instructor. This work presents approaches to automize the generation of DFA minimization tasks.
	\end{abstract}

	\renewcommand{\abstractname}{Generierung von DFA Minimierungsproblemen\\-\\Zusammenfassung}
	\begin{abstract}
		Zusammenfassung$\ldots$
	\end{abstract}
	
	%\chapter*{Acknowledgements}
	%I want to thank...
	
	\renewcommand{\contentsname}{Table of Contents}
	\tableofcontents
	
	% !TeX spellcheck = en_US

\chapter{Introduction}

\begin{itemize}
	\item study computer science
	\item theoretical informatics
	\item automata theory
	\item value of this theory
	\item typical topics, why typical
	\item why automation
\end{itemize}

This work lays out the theory for a program solving this task. As a consequence, parameters, which are sensible as user input, will be incorporated in problem definitions.
In addition, when evaluating possible algorithms, we will take their usability in a practical use case into account.
Furthermore additional theory will be discussed, to enhance usability.

\section{Preliminaries}

We start with defining preliminary theoretical foundations.

\subsection{Deterministic Finite Automatons}

A 5-tuple $A = (Q, \Sigma, \delta, s, F)$ with $Q$ being a finite set of \emph{states}, $\Sigma$ a finite set of \emph{alphabet symbols}, $\delta \colon\ Q \times \Sigma \to Q$ a \emph{transition function}, $s \in Q$ a \emph{start state} and $F \subseteq Q$ \emph{final states} is called \emph{deterministic finite automaton} (DFA)~\cite[p. 46]{hopcroft01}. From now on $\mathcal{A}$ shall denote the set of all DFAs.

We say $\delta(q,\sigma) = p$ is a transition from $q$ to $p$ using symbol $\sigma$. We define the \emph{extended transition function} $\delta^* : Q \times \Sigma^* \to Q$ of a DFA $A = (Q, \Sigma, \delta, s, F)$ as:
\begin{itemize}
	\item $\delta^*(q,\varepsilon) = q$
	\item $\delta^*(q,w\sigma) = \delta(\delta^*(q,w),\sigma)$ for all $q \in Q$, $w \in \Sigma^*$, $\sigma \in \Sigma$
\end{itemize}
Then, the \emph{language} of that DFA is defined as $L(A) = \{\ w\ |\ \delta^*(w) \in F\ \}$~\cite[pp. 49-50. 52]{hopcroft01}.

Given a state $q \in Q$. We call all transitions $\delta(q', \sigma) = q$ \emph{ingoing} transitions of $q$. All transitions $\delta(q, \sigma) = q'$ are called \emph{outgoing} transitions of $q$. If a transition is of the form $\delta(q, \sigma) = q$, then we say that $q$ has a \emph{loop}.

\begin{definition}\label{ch:1:unreachable-states}
	We say a state $q$ is \emph{(un-)reachable} in an DFA $A$, iff there is (no) a word $w \in \Sigma^*$ such that $\delta^*(s, w) = q$.
\end{definition}

A DFA is called \emph{complete} iff for all states, every symbol of the alphabet is used on an outgoing transition: $\forall q\in Q\colon \forall\sigma\in\Sigma\colon \exists p\in Q\colon \delta(q,\sigma) = p$. Note, that every incomplete DFA can be converted to a complete one by adding a so called \emph{dead state}~\cite[p. 67]{hopcroft01}. The resulting automaton has the same language. We will from now on only work with complete DFAs.

\begin{figure}
	\includegraphics[width=\linewidth]{images/dfa.png}
	\caption{An example DFA and its properties.}
	\label{fig:dfa}
\end{figure}

\subsection{Minimal DFAs}

This section closely follows~\cite[pp. 42-45]{schoening01}. We call a DFA $A$ \emph{minimal}, if there exists no other automaton with the same language using less states. With $\mathcal{A}_{min}$ we shall denote the set of all minimal DFAs.

The \emph{Nerode-relation} $\equiv_L\ \subseteq\ \Sigma^* \times \Sigma^*$ of a language $L$ with alphabet $\Sigma$ is defined as follows:
\begin{displaymath}
	x \equiv_L y\ \Leftrightarrow_{def}\ \forall z\in\Sigma^*\colon (xz\in L \Leftrightarrow yz\in L)
\end{displaymath}
The Nerode-relation of a DFA $A$ is the the Nerode-relation of its language: $\equiv_{L(A)}$. If the context makes it clear, than we will shorten the notation of a equivalence class $[x]_{\equiv_L}$ with $[x]$.

The \emph{equivalence class automaton} $A_L = (Q_L, \Sigma_L, \delta_L, s_L, F_L)$ to a regular language $L$ with alphabet $\Sigma$ is defined as follows:
\begin{itemize}
	\item $Q_L = \{\ [x]\ |\ x \in \Sigma^*\ \}$
	\item $\Sigma_L = \Sigma$
	\item $\delta_L([x], \sigma) = [x\sigma],\ \forall x\in\Sigma^*,\ \forall\sigma\in\Sigma$
	\item $s = [\varepsilon]$
	\item $F = \{\ [x]\ |\ x \in L\ \}$
\end{itemize}
\begin{theorem}
	Given a language $L$, then the equivalence class automaton $A_L$ is minimal.
\end{theorem}

\subsection{Isomorphy of DFAs}

Given two DFAs $A_1 = (Q_1, \Sigma_1, \delta_1, s_1, F_1)$ and $A_2 = (Q_2, \Sigma_2, \delta_2, s_2, F_2)$. We say $A_1$ and $A_2$ are \emph{isomorph} ($A_1 \cong A_2$), iff:
\begin{itemize}
	\item $\Sigma_1 = \Sigma_2$ and
	\item there exists a bijection $\pi\colon Q_1 \to Q_2$ such that:
	
	$\pi(s_1) = s_2$
	
	$\forall q\in Q_1\colon (q\in F_1 \Longleftrightarrow \pi(q)\in F_2)$
	
	$\forall q\in Q_1\colon \forall\sigma\in\Sigma\colon (\pi(\delta_1(q,\sigma))=\delta_2(\pi(q),\sigma))$
\end{itemize}
\begin{theorem} \textnormal{\cite[p. 45]{schoening01}} 
	Every minimal DFA is unique except for isomorphy.
\end{theorem}
\begin{corollary}\label{ch:1:cor:all-min-dfa-ism}
	Every minimal DFA $A$ is isomorph to its corresponding equivalence class automaton $A_{L(A)}$.
	\gregor{All min. DFAs are ism. to each other, including A\_L}
\end{corollary}

\subsection{Duplicate states}

\begin{definition}[Duplicate States]\cite[p. 154]{hopcroft01}
	Two states $q_1, q_2 \in Q$ of a DFA $A = (Q, \Sigma, \delta, s, F)$ are called \emph{duplicates} of each other, iff $d_A(q_1, q_2)$ is true, whereas
	\begin{displaymath}
	q_1\ d_A\ q_2\ \Leftrightarrow_{def}\ \forall z \in \Sigma^* \colon\ (\delta^*(q_1, z) \in F \Leftrightarrow \delta^*(q_2, z) \in F)
	\end{displaymath}
\end{definition}
\noindent Note that the relation $d_A$ is indeed an equivalence relation.

\subsection{The minimization algorithm}

This minimization algorithm requires a complete DFA and works in four major steps, removing essentially states in such a way, that no unreachable and no duplicate states are left.
\begin{enumerate}
	\item Compute all unreachable states via breadth-first search for example.
	
	\vspace{0.2cm}
	\begin{algorithmic}[1]
		\Function{ComputeUnreachableStates}{$A$}
			\State $U \gets Q \setminus \{s\}$	\Comment{undiscovered states}
			\State $O \gets \{s\}$				\Comment{observed states}
			\State $D \gets \{\}$				\Comment{discovered states}
			\While {$|O| > 0$}
				\State $N \gets \{\ p\ | \ \exists q \in O\ \sigma \in \Sigma \colon\ \delta(q, \sigma) = p\ \land\ p \notin O \cup D\ \}$
				\State $U \gets U \setminus N$
				\State $D \gets D \cup O$
				\State $O \gets N$
			\EndWhile
			\State \Return $U$
		\EndFunction
	\end{algorithmic}

	\item Remove all unreachable states and their transitions.
	
	\vspace{0.2cm}
	\begin{algorithmic}[1]
		\Function{RemoveUnreachableStates}{$A, U$}
			\For {$q$ \textbf{in} $U$}
				\If {$q \in F$}
					\State $F \gets F \ \{q\}$
				\EndIf
				\State $\delta \gets \delta \setminus \{\ ((q_1, \sigma), q_2) \in \delta\ |\ q_1 = q \lor q_2 = q\ \}$
			\EndFor
			\State \Return $A$
		\EndFunction
	\end{algorithmic}

	\item Compute all non-duplicate states ($\neg d_A(p, q)$) via the \MinMark-algorithm.
	
	\vspace{0.2cm}
	\begin{algorithmic}[1]
		\Function{\MinMark}{$A$}
		\State $M \gets \{ (p,q), (q,p)\ |\ p \in F, q \notin F \}$
		\Do
			\State $M' \gets \{ (p,q)\ |\ (p,q) \notin M \land \exists \sigma \in \Sigma \colon (\delta(p,\sigma), \delta(q,\sigma)) \in M \}$
			\State $M \gets M \cup M'$
		\doWhile {$M' \neq \emptyset$}
		\State \Return $M$
		\EndFunction
	\end{algorithmic}

	\item Merge all duplicate state pairs, which are exactly those, that are not in $\neg d_A$.
	
	\vspace{0.2cm}
	\begin{algorithmic}[1]
		\Function{\MinMerge}{$A, \neg d_A$}
			\State compute $d_A$
			\While {$d_A \neq \emptyset$}
				\State $(p,q) \in d_A$
				\State $d_A \gets d_A \setminus \{ (p,q) \}$
				\If {$p \neq q$}
					\State exchange all occurrences of $q$ in $A$ and $d_A$ by $p$
				\EndIf
			\EndWhile
			\State \Return $A$
		\EndFunction
	\end{algorithmic}
\end{enumerate}
\begin{theorem}
	The minimization algorithm computes a minimal DFA to its input DFA.
\end{theorem}
\noindent The definition of this DFA minimization algorithm is inspired by Schöning~\cite[p. 46]{schoening01}.

When looking at \MinMark, one notes, that it computes distinct subsets of $Q \times Q$ on the way. Indeed, one could write the algorithm in such a way, that these subsets are explicitly computed in form of a function $m\colon\mathbb{N}\to\mathcal{P}(Q\times Q)$:
\vspace{0.2cm}
\begin{algorithmic}[1]
	\Function{$m$-MinimizationMark}{$(Q, \Sigma, \delta, s, F)$}
	\State $i \gets 0$
	\State $m(0) \gets \{ (p,q), (q,p)\ |\ p \in F, q \notin F \}$
	\Do
		\State $i \gets i + 1$
		\State $m(i) \gets \{ (p,q)\ |\ (p,q) \notin \bigcup{m(\cdot)} \land \exists \sigma \in \Sigma \colon (\delta(p,\sigma), \delta(q,\sigma)) \in m(i-1) \}$
	\doWhile {$m(i) \neq \emptyset$}
	\State \Return $\bigcup{m(\cdot)}$
	\EndFunction
\end{algorithmic}
\vspace{0.2cm}
Using this redefinition, we can easier refer to the state pairs marked in a certain iteration. We will use both variants in exchange.

We will denote the number of iterations done by \MinMark\ on an DFA $A$ as $\mmD(A)$. Note that $\mmD(A) = \max n \in \mathbb{N}\ |\ m(n) \neq \emptyset$.


\section{Requirements analysis}

\subsection{Example of a DFA minimization task for students}

\begin{itemize}
	\item present a typical task and its solution (text, image, table)
	\item name its elements
	\item clarify two parts: solution, task
	\item clarify four parts from task to solution: find unreachables, delete them, find duplicates, merge them
	\item what are the difficulties of this tasks?
\end{itemize}

We will call the minimal automaton \emph{solution DFA} ($A_{sol}$) and its extension with duplicate and unreachable states \emph{task DFA} ($A_{task}$).

heuristic:
\begin{itemize}
	\item $h \colon \mathcal{A} \times \mathcal{A} \to \mathbb{R^+}$
	\item $h(A_{min}, A_{task}) = studentfriendliness$
\end{itemize}

\subsection{Definition and evaluation of possible requirements}

\label{ch:1:determined-requirements}
Dismissed:
\begin{itemize}
	\item $h(A_{sol}, A_{task}) = |Q_{task}|\ /\ |Q_{sol}|$
	\item number of transitions
	\item max degree of a node (Why not this?)
	\item Does GraphViz have a heuristic?
\end{itemize}
Accepted solution DFA criteria:
\begin{itemize}
	\item[->] minimal
	\item[->] number of states
	\item[->] number of \MinMark\ iterations ($\mmD(A_{sol})$)
	\item[->] alphabet size
	\item[->] number of accepting states
	\item[->] planarity (can be checked in $O(|Q_{sol}|)$)
	\item[->] $A_{sol}$ is unused (regarding all previously generated solution DFAs)
	
	\begin{definition}[Unused DFAs] \label{ch:1:unused-dfa}
		A DFA $A$ is \emph{unused} regarding a set of \emph{used DFAs} $U$, if $A$ is not isomorph to any DFA in $U$.
	\end{definition}
\end{itemize}
Accepted task DFA criteria:
\begin{itemize}
	\item[->] $L(A_{sol}) = L(A_{task})$
	\item[->] $\mmD(A_{sol}) = \mmD(A_{task})$
	\item[->] number of duplicate states
	\item[->] number of unreachable states
	\item[->] alphabet size
	\item[->] planarity (can be checked in $O(|Q_{task}|)$)
	\item[->] completeness (for \MinMark-algorithm to work)
\end{itemize}

\section{Approach and general algorithm}

In this work we will first build the solution DFA (step 1), and - based on that - the task DFA by adding duplicate and unreachable states (step 2). Both steps will fulfill all criteria chosen above and are covered in depth in chapter~\ref{ch:2} respectively chapter~\ref{ch:3}.

It follows that $\mmD$ and $L$ of both DFAs will be set when building $A_{sol}$. As a consequence we need to ensure that adding duplicate and unreachable state does neither change $\mmD(A_{task})$ nor $L(A_{task})$ in comparison to $A_{sol}$. We will do this during the discussion of step 2.

Here follow problem definitions for the two steps, which specify all needed informations. \gregor{Hidden formulation here} %The first problem is lain out in a way, such that it does

\begin{definition}[BuildNewMinimalDFA] $ $
	\begin{description}
		
		\item[Given:] $ $
		
		$q, a, f, m_{min}, m_{max} \in \mathbb{N},$
		
		$p \in \{0,1\}$
		\item[Request:] $ $
		
		Let $A_{sol} = (Q, \Sigma, \delta, s, F)$ be a DFA, such that
		
		\qquad $|Q|=q$, $|\Sigma|=a$, $|F|=f$,
		
		\qquad $m_{min} \le \mmD(A_{sol}) \le m_{max}$,
		
		\qquad $A_{sol}$ is planar iff $p = 1$ and
		
		\qquad the language of $L(A)$ is unequal to any DFA used before.
		
		Return $A_{sol}$, if it exists, $\bot$ otherwise.
	\end{description}
\end{definition}

\begin{definition}[ExtendMinimalDFA] $ $
	\begin{description}
		
		\item[Given:] $ $
		
		$A_{sol} = (Q, \Sigma, \delta, s, F) \in \mathcal{A}_{min},$
		
		$p \in \{0,1\},$
		
		$d, u \in \mathbb{N}$
		\item[Request:] $ $
		
		A DFA $A_{task} = (Q', \Sigma', \delta', s', F')$ with reachable duplicate states $q_1 \ldots q_d$ and unreachable states $p_1 \ldots p_u$, such that
		
		$Q = Q' \cup \{ q_1, \ldots q_d, p_1 \ldots p_u \}$,
		
		$\Sigma = \Sigma'$, $s = s'$,
		
		$F \subseteq F'$,
		
		$A_{task}$ is planar iff $p = 1$,
		
		$L(A_{sol}) = L(A_{task})$ and $\mmD(A_{sol}) = \mmD(A_{task})$.
	\end{description}
\end{definition}

\noindent The main algorithm will then simply be:
\vspace{0.2cm}
\begin{algorithmic}[1]
	\Function{GenerateDFAMinimizationProblem}{$q, a, f, m_{min}, m_{max}, p_1, p_2, d, u$}
	\State $A_{sol} \gets \textsc{BuildNewMinimalDFA}(q, a, f, m_{min}, m_{max}, p_1)$
	\State $A_{task} \gets \textsc{ExtendMinimalDFA}(A_{sol}, p_2, d, u)$
	\State \Return $A_{sol}, A_{task}$
	\EndFunction
\end{algorithmic}



	% !TeX spellcheck = en_US

\chapter{Building minimal DFAs}

We want an algorithm for DFA generation that fulfills the following conditions:
\begin{itemize}
	\item the resulting DFA is minimal
	\item we can control the number of states of the resulting DFA via parameter
	\item we can control $\mmD$ of the resulting DFA via parameter
\end{itemize}

\section{Using trial and error}

\subsection{General algorithm}

Dividing the search space, looking only at DFAs with correct alphabetSize, numberOfStates, numberOfAcceptingStates

\vspace{0.2cm}
\begin{algorithmic}[1]
	\Function{BuildNewMinimalDFA\ }{
		
$q, a, f, m_{min}, m_{max} \in \mathbb{N},$

$p \in \{0,1\}$

}
		\While {True}
			\State generate DFA $A_L$ with $|Q|, |\Sigma|, |F|$ matching $q, a, f$
			
			\If {$A_L$ not minimal}
				\State \textbf{continue}
			\EndIf
			
			\If {$p = 1$ \textbf{and} $A_L$ is not planar}
				\State \textbf{continue}
			\EndIf
			
			\If {another DFA with the same language was used before}
				\State \textbf{continue}
			\EndIf
			
			\State\Return $A_L$
		\EndWhile
	\EndFunction
\end{algorithmic}
\vspace{0.2cm}

\subsection{Ensuring $A_{test}$ is minimal and $\mmD(A_{test})$ is correct}

has unreachable states, has duplicate states, computing $\mmD(A)$ as a side effect

\vspace{0.2cm}
\begin{algorithmic}[1]
	\Function{BuildNewMinimalDFA\ }{$q, a, f, m_{min}, m_{max}, p$}
		\While {True}
			\State generate DFA $A_L$ with $|Q|, |\Sigma|, |F|$ matching $q, a, f$
			
			\If {$A_L$ has duplicate or unreachable states}
				\State \textbf{continue}
			\EndIf
			
			\ldots
		\EndWhile
	\EndFunction
\end{algorithmic}
\vspace{0.2cm}

\subsection{Ensuring $A_{test}$ is planar}

look up library algorithm, cite some papers

dfa specific planarity test?
use planarity test information for better drawing?

\subsection{Ensuring $A_{test}$ is unused}

\paragraph*{DB1 - Found Minimal DFAs}

\begin{tabular}{c c c c c c}
	dfa\_id & |Q| & minDepth & |F| & |$\Sigma$| & used
\end{tabular}\\
dfa\_id is the encoded DFA. This table saves already found minimal DFAs.

\vspace{0.2cm}
\begin{algorithmic}[1]
	\Function{BuildNewMinimalDFA\ }{$q, a, f, m_{min}, m_{max}, p$}
		\State $l \gets$ load based on all parameters
		\While {True}
			\State generate DFA $A_L$ with $|Q|, |\Sigma|, |F|$ matching $q, a, f$
			
			\If {$A_L$ not minimal}
				\State \textbf{continue}
			\EndIf
			
			\If {$p = 1$ \textbf{and} $A_L$ is not planar}
				\State \textbf{continue}
			\EndIf
			
			\If {$A_L$ isomorph to any DFA in $l$ matching with $q, a, f, m_{min}, m_{max}, p$}
				\State \textbf{continue}
			\EndIf
			
			\State save $l \cup A_L$
			\State\Return $A_L$
		\EndWhile
	\EndFunction
\end{algorithmic}
\vspace{0.2cm}

\subsection{Option 1: Generating $A_{test}$ via Randomness}

\subsection{Option 2: Generating $A_{test}$ via Enumeration}

Using this method we have a lot of freedom. The biggest problem we are approaching here is termination.

\paragraph*{DB2 - Enumeration Progress}

\begin{tabular}{c c c c c}
	|Q| & |$\Sigma$| & f & t & finished
\end{tabular}\\
f and t are the encoded final states respectively transitions enumeration progresses for the given |Q| and |$\Sigma$| (and f in case of t). With this table the enumeration room is split into smaller pieces.

\paragraph*{The enumeration algorithm}

\vspace{0.2cm}
\begin{algorithmic}[1]
	\Function{BuildNewMinimalDFA\ }{$q, a, f, m_{min}, m_{max}, p$}
		\State $l \gets$ load based on all parameters
		\State $e \gets$ load enumeration progress for $q, a, f, p$
		\While {True}
			\If {$e$ is finished}
				\State save $e$
				\State\Return $\bot$
			\EndIf
			
			\State $A_L \gets$ next DFA based on $e$
			
			\If {$A_L$ not minimal}
				\State \textbf{continue}
			\EndIf
			
			\If {$p = 1$ \textbf{and} $A_L$ is not planar}
				\State \textbf{continue}
			\EndIf
			
			\If {$A_L$ isomorph to any DFA in $l$ matching with $q, a, f, m_{min}, m_{max}, p$}
				\State \textbf{continue}
			\EndIf
			
			\State save $e$
			\State save $l \cup A_L$
			\State\Return $A_L$
		\EndWhile
	\EndFunction
\end{algorithmic}
\vspace{0.2cm}

\subsection{Ideas for more efficiency}

incrementing final state binary faster in enum-alternative

speed up isomorphy test

rewrite everything in C

solve P vs NP

\section{Alternative approach: Building $m(i)$ bottom up}

Build $m$ from $m$-\MinMark\ iteratively. (Why would this basically result in running \MinMark\ all the time?)

	% !TeX spellcheck = en_US

\chapter{Extending minimal DFAs} \label{ch:3}

We firstly define a formal problem for extending a minimal DFA $A_{sol}$ to a task DFA $A_{task}$ based on our requirements analysis (see~\ref{ch:1:determined-requirements}):
\begin{definition}[ExtendMinimalDFA] $ $ \\
	$ $ \vspace{-0.cm} \\
	\noindent $\underline{\emph{Given:}}$
	\vspace{-0.2cm}
	\begin{align*}
	A_{sol} = (Q, \Sigma, \delta, s, F) \in \mathfrak{A}_{min}\ \ \ & \emph{solution DFA} \\
	\mathcal{Q}_{eq} \in \mathbb{N}\ \ \ & \emph{number of states creating equivalent state pairs} \\
	\mathcal{Q}_{unr} \in \mathbb{N}\ \ \ & \emph{number of unreachable states} \\
	p \in \{0,1\}\ \ \ & \emph{planarity-bit} \\
	c \in \{0,1\}\ \ \ & \emph{completeness-bit}
	\end{align*}
	\noindent $\underline{\emph{Task:}}$ \emph{Compute, if it exists, a task DFA $A_{task}$ with}
	\begin{itemize}
		\item $Q_{task} = Q_{sol} \cup \{ r_1, \ldots, r_{\mathcal{Q}_{eq}}, u_1, \ldots, u_{\mathcal{Q}_{unr}} \}$
		\item $r_1, \ldots, r_{\mathcal{Q}_{eq}}$ \emph{each creating an equivalent state pair}
		\item $u_1, \ldots, u_{\mathcal{Q}_{unr}}$ \emph{unreachable}
		\item $\Sigma_{task} = \Sigma_{sol}$, $s_{task} = s_{sol}$, $F_{task} \subseteq F_{sol}$
		\item $A_{task}$ \emph{being planar iff} $p = 1$
		\item $A_{task}$ \emph{being complete iff} $c = 1$
		\item $A_{sol}$ \emph{being isomorph to} $\MinAlg(A_{task})$
	\end{itemize}
\end{definition}
\noindent In order to fulfill these requirements we will deduce for both kinds of states how they may be added by examining their desired properties. We will show for the action of adding equivalent states, that this does not change a DFAs $\mmD$-value.

\section{Creating equivalent state pairs}

Step 3 and 4 of the minimization algorithm are concerned with detection and elimination of equivalent state pairs. We now want to add states $r_1,\ldots,r_{\mathcal{Q}_{eq}}$ to a DFA $A_{sol}$, gaining $A_{re}$ with $Q_{re} = Q_{sol} \cup \{r_1,\ldots,r_{\mathcal{Q}_{eq}}\}$, such that each of these states is equivalent to a state in $A_{re}$. Note that, for reasons of clarity, we are going to abbreviate from now on $A_{re} = A$, $Q_{re} = Q$, $\sim_{A_{re}} = \sim_A$ etc.

%At this point we notice, that $A_{sol}$ is isomorph to the $\sim$-equivalence automaton (see def.~\ref{ch:1:sim-eq-dfa}). We can think 

Consider the properties $r_1,\ldots,r_{\mathcal{Q}_{eq}}$ must have. They are equivalent to states $o_1,\ldots,o_{\mathcal{Q}_{eq}}$ of $A$.
\[
\exists r_1,\ldots,r_{\mathcal{Q}_{eq}} \in Q\colon\ \exists o_1,\ldots,o_{\mathcal{Q}_{eq}} \in Q\colon\ \forall i \in [1,\mathcal{Q}_{eq}] \colon\ r_i \sim_A o_i
\]
But we know also and in particular, that each of them is equivalent a state $e$ of $A_{sol}$.
\[
	\exists r_1,\ldots,r_{\mathcal{Q}_{eq}} \in Q\colon\ \forall i \in [1,\mathcal{Q}_{eq}] \colon\ \exists e \in Q_{sol}\colon\ r_i \sim_A e
\]
In our algorithm, we will choose the state $e$ for each state we add.

\subsection{Adding outgoing transitions}

Regarding the outgoing transitions of any $r_i$ equivalent to a state $e$, we are directly restricted by the relationship $\forall \sigma \in \Sigma \colon [\delta(r_i, \sigma)]_{\sim_A} = [\delta(e, \sigma)]_{\sim_A}$. Thus, when adding some $r_i$, we have to choose for each symbol $\sigma \in \Sigma$ at exactly one transition (completeness requirement for $A$) from the following set:
\[
	O_{e,\sigma} = \{\ ((r_i, \sigma), q)\ |\ q \in [\delta(e, \sigma)]_{\sim_A}\ \}
\]
Since the solution DFA is complete and since every here added state gets a transition for every alphabet symbol, we know that every $O_{e,\sigma} \neq \emptyset$.

\gregor{Why does this not affect the eq. class of any other state?}

\subsection{Adding ingoing transitions}

First of all, we know, that $r_i$ is reachable, since every state of $A$ must be reachable, so we need to give $r_i$ at least one ingoing transition. Doing this, we have to ensure, that any state $q$, that gets such an outgoing transition to $r_i$ remains in its $\sim$-equivalence class.
	
Thus a fitting state $q$ has to have a transition to some state in $[r_i]_{\sim_A} = [e]_{\sim_A}$ already. So, given a state $q$ with $\delta(q, \sigma) = p$ and $p \in [e]_{\sim_A}$, we can set $\delta(q, \sigma) = r_i$ and thus ``steal'' $q$ its ingoing transition.

We see here, that $q$ must have at least $2$ ingoing transitions, else it would become unreachable. Thus we summarize:
\[
    I_e = \{\ ((q, \sigma), p)\ |\ \delta(q, \sigma) = p \land p \in [e] \land d^-(p) \geq 2\ \}
\]
Choose at least one $((q, \sigma), p) \in I_e$, remove $((q, \sigma), p)$ from $\delta$ and add $((q, \sigma), r_i)$. 

These finding lead us to a general requirement regarding the choice of a state $e$ for an $r_i$: The equivalence class of any $e$ has to contain at least one state with at least $2$ ingoing transitions (see fig.~\ref{fig:dfa_create_equivalent_states}). We establish the following notion to pin down this restriction:
\[
	duplicatable(q) \Leftrightarrow_{def} (\exists p \in [q]_{\sim_A}\colon |d^-(p)| \geq 2)
\]
The number of duplicatable states in any accessible DFA $A$ is $0$ for $|\Sigma| \leq 1$ (due to the restriction $|d^-(p)| \geq 2$) and greater than $0$ for $|\Sigma| > 1$ due to the pigeonhole principle: An accessible complete DFA has $|Q||\Sigma|$ transitions which have to be spread across $|Q|$ states.
\begin{figure}
	\includegraphics[width=\linewidth]{images/dfa_create_equivalent_states.png}
	\caption{If an equivalence class (here denoted by the states in the dashed area) contains a state with 2 or more ingoing transitions (in this case $p$), then a state equivalent to any of the classes states may be added. Here $r$ is equivalent to $o$ and is ``stealing'' the ingoing transition $\delta(q, a)$ from $p$.}
	\label{fig:dfa_create_equivalent_states}
\end{figure}

\subsection{The algorithm}

\vspace{0.2cm}
\begin{spacing}{1}
\begin{algorithmic}[1]
	\Function{CreateEquivalentStatePairs}{$A, \mathcal{Q}_{eq}$}
    \State $Q \gets Q_{sol}$
    \State $\delta \gets \delta_{sol}$
    \State $F \gets F_{sol}$
	\State $K \gets \{\ \{q\}\ |\ q \in Q\ \}$ \Comment{tracks the equivalence classes of $A$}
	\State $k(q) = C$ such that $q \in C$ and $C \in K$ \Comment{returns the equivalence class to $q$}
	\State $in(q) = |d^-(q)|$ for all $q \in Q$ \Comment{tracks the number of ingoing t.}
	
    \For {$i$ \textbf{in} $[1,\mathcal{Q}_{eq}]$}
		\For {$q$ \textbf{in} $Q$} \Comment{find a duplicatable state $e$}
			\If {$in(q) \geq 2$}
				\State $e \gets$ random chosen state from $k(q)$
				\State \textbf{break}
			\EndIf
		\EndFor
		
        \State Add $r_i$ to $Q$ \Comment{create to $e$ equivalent state $r_i$}
		\State Add $r_i$ to $k(e)$
		
		\For {$\sigma$ \textbf{in} $\Sigma$} \Comment{add $d^+(r_i)$}
			\State $\delta(r_i, \sigma) =$ random chosen state from $k(\delta(e, \sigma))$
		\EndFor
		
		\State $P \gets \{\ ((s, \sigma), t) \in \delta\ |\ t \in k(e),\ in(t) \geq 2\ \}$ \Comment{add $d^-(r_i)$}
		\State $C \gets$ random nonempty subset of $P$
		\For {$((s, \sigma), t)$ \textbf{in} $C$}
			\State $in(t) \gets in(t) - 1$
			\State $in(r_i) \gets 1$
            \State $\delta(s, \sigma) = r_i$
		\EndFor
	\EndFor
    \State \Return $(Q, \Sigma_{sol}, \delta, s_{sol}, F)$
	\EndFunction
\end{algorithmic}
\end{spacing}
\vspace{0.2cm}

\subsection{Creating equivalent state pairs does not change $\mmD$} \label{ch:3:sec-D-proof}

To prove this statement, we will prove two minor propositions first. In this context we will call a word $w$ \emph{distinguishing word of $p,q$}, iff $distinguishes(w, q_0, q_1)$ whereas
\[
    distinguishes_A(w, q_0, q_1) \Leftrightarrow (\delta^*(p,w) \in F \Leftrightarrow \delta^*(q,w) \notin F) \Leftrightarrow p \not\sim_A q
\]

\begin{lemma}\label{ch:3:semantics-of-m(n)}
    In the context of \CompDist\ the following is true:
    \[
        (p,q) \in m(n) \Longleftrightarrow \exists w\in\Sigma^*\colon (|w| = n\ \land distinguishes_A(w, p, q))
    \]
\end{lemma}

\begin{proof}
	See TI-Lecture ch. 4 ``Minimization'' p. 18.
\end{proof}

\begin{lemma}\label{ch:3:semantics-of-D(A)}
    The longest distinguishing word $w$ of a DFA $A$ has length $\mmD(A)-1$.
%    Formally:
%	\begin{multline*}
%		\mmD(A) =\ n \Rightarrow \\
%		n = \max_{n \in \mathbb{N}}\ \ \exists p, q \in Q\ \ \exists w \in \Sigma^* \colon \hspace{3cm} \\
%		(|w| = n - 1\ \land distinguishes_A(w, p, q))
%	\end{multline*}
\end{lemma}

\begin{proof}
	Via direct proof. Assume $m$-\CompDist(A) has done $n$ iterations (so $\mmD(A) = n$). We then know, that
	\begin{enumerate}
		\item $\forall i \in [0,n-1]\colon m(i) \neq \emptyset$
		\item $\forall i > n-1\colon m(n)= \emptyset$ \gregor{fix this}
	\end{enumerate}
	$m$-\CompDist(A) terminates iff $m(i) = \emptyset$. If the first point would not hold, then the algorithm would have stopped before.
	
	Since the algorithm did $n$ iterations, the internal variable $i$ must be $n$ at the end of the last iteration. The terminating condition is $m(i) \neq \emptyset$; thus follows the second point.
    
    We prove that then there exists a distinguishing word of length $n-1$, but the existence of a longer word leads to a contradiction. Recall lemma~\ref{ch:3:semantics-of-m(n)}:
    \[
        (p,q) \in m(n) \Longleftrightarrow \exists w\in\Sigma^*\colon (|w| = n\ \land distinguishes(w, p, q))
    \]

	% a possible word per definition of D(A), m(i) and lemma
	
	\noindent Following this lemma and point 1, we can deduce that there exists at least one distinguishing word $w$ with $|w| = n-1 = \mmD(A)-1$ for some $p,q \in Q$.
	
	% There is no word longer than that
	
	There cannot be any distinguishing word $w'$ with $|w'| = k > n-1$ for any two states $p',q'\in Q$ fulfilling this property. Following the lemma again, $m(k), k > n-1$ would be non-empty, which is contradictory to point 2.
\end{proof}

\begin{lemma}\label{ch:3:lem:disting-trans}
    \[
    distinguishes_A(w, p, q) \land q \sim_A q' \Rightarrow distinguishes_A(w, p, q')
    \]
\end{lemma}

\begin{proof}
    Via direct proof.
    \begin{align*}
        &\ distinguishes_A(w, p, q) \land q \sim_A q' \\
        \Rightarrow &\ (\delta^*(p,w) \in F \Leftrightarrow \delta^*(q,w) \notin F) \land (\forall z \in \Sigma^* \colon\ (\delta^*(q, z) \in F \Leftrightarrow \delta^*(q', z) \in F)) \\
        \Rightarrow &\ (\delta^*(p,w) \in F \Leftrightarrow \delta^*(q,w) \notin F) \land (\delta^*(q, w) \in F \Leftrightarrow \delta^*(q', w) \in F) \\
        \Rightarrow &\ (\delta^*(p,w) \in F \Leftrightarrow \delta^*(q',w) \notin F) \\
        \Rightarrow &\ distinguishes_A(w, p, q')
    \end{align*}
\end{proof}

\begin{theorem}
	Creating equivalent state pairs in a minimal DFA $A$ does not increase the number of iterations when the \CompDist-algorithm is applied on it.
\end{theorem}

\begin{proof}
	Per contradiction. Let us assume there were $n$ states $r_0, \ldots, r_n$ added to a given minimal DFA $A = (Q, \Sigma, \delta, s, F)$ resulting in a DFA $A' = (Q', \Sigma, \delta', s, F')$ such that:
	\begin{itemize}
		\item $Q' = Q \cup \{ r_1, \ldots, r_n \}$
		\item $\forall i \in [1,n] \colon\ \exists q \in Q\colon\ r_i \sim_{A'} q$
        \item $\mmD(A) < \mmD(A')$
	\end{itemize}
    According to lemma~\ref{ch:3:semantics-of-D(A)} the longest distinguishing word $w$ of $A$ has length $|w| = \mmD(A) - 1$, while its counterpart $w'$ in $A'$ has length $\mmD(A') - 1$. Consequently $|w| < |w'|$.
    
    There exist states $p',q' \in Q'$ such that $w'$ distinguishes $p',q'$. We differentiate between three cases regarding belonging of $p',q'$.
    
    \paragraph*{Both $p',q'$ in $Q$:} Since the longest distinguishing word any state pair in $Q$ can have is guaranteed shorter than $w'$, we may conclude that $p',q' \in Q$ and $p',q'$ have $w'$ as distinguishing word is a contradiction. \lightning
    
    \paragraph*{One of $p',q'$ in $Q$, one in $Q'\setminus Q$:} W.l.o.g. $p' \in Q$, $q' \in Q'\setminus Q$. We then know, that $q' \in \{r_1, \ldots, r_n\}$. This implies that $q' \sim_{A'} q$ for an $q \in Q$.
    
    By lemma~\ref{ch:3:lem:disting-trans} $w'$ is a distinguishing word for $p',q$. But this is contradictory to $p',q \in Q$. \lightning
        
    \paragraph*{None of $p',q'$ in $Q$:} Since $p',q'$ both have to be in $Q'\setminus Q = \{r_1, \ldots, r_n\}$, we can find states $p, q \in Q$ equivalent to $p',q'$. These states would be distinguished by $w'$, which is contradictory again. \lightning 
\end{proof}

\section{Adding unreachable states}

From step 1 of the minimization algorithm we can deduce how to add unreachable states. These can easily be added to a DFA by adding non-start states with no ingoing transitions (see def.~\ref{ch:1:unreachable-states}). Number and nature of outgoing transitions may be arbitrary.

\vspace{0.2cm}
\begin{algorithmic}[1]
	\Function{AddUnreachableStates\ }{$A, \mathcal{Q}_{unr}, c$}
	\For {$\mathcal{Q}_{unr}$ \textbf{times}}
		\State $q \gets \max Q + 1$
		\State $Q \gets Q \cup \{ q \}$
        \State $outSymbols \gets c = 1\ ?\ \Sigma\ :\ $ random subset of $\Sigma$
		\State $R \gets$ random chosen sample of $|outSymbols|$ states from $Q \setminus \{q\}$
		\For {$\sigma$ \textbf{in} $outSymbols$}
			\State $q' \in R$
			\State $R \gets R \setminus \{q'\}$
			\State $\delta \gets \delta \cup \{ ((q, \sigma), q') \}$
		\EndFor
	\EndFor
	\State \Return $A$
	\EndFunction
\end{algorithmic}
\vspace{0.2cm}

%\noindent We have to ensure, that this algorithm does not induce changes in the language.
%\begin{lemma}
%	Adding unreachable states to a DFA does not change its language.
%\end{lemma}
%\begin{proof}
%	{\setlength{\parindent}{0pt}
%		Remember that the language of a DFA $A = (Q, \Sigma, \delta, s, F)$ is defined as $L(A) = \{\ w\ |\ w \in \Sigma^* \ \}$. For any unreachable state $q$ there exists no word $v \in \Sigma^*$ such that $\delta^*(s,v) = q$. Thus such a state cannot be the cause for any word to be in $L(A)$.
%	}
%\end{proof}
%
%\noindent The question whether adding unreachable states to a DFA changes $\mmD$-value is irrelevant. This is because in the context of the minimization algorithm, unreachable states are eliminated before the \CompDist-algorithm is applied on the task DFA.

	% !TeX spellcheck = en_US

\chapter{Conclusion}

We close this work with a summary and a short lookout.

\paragraph*{Summary.}

In this work we discussed and implemented methods to automatically generate exercises for students that consist of a DFA $A_{task}$ which has to be minimized. We focused on the minimization algorithm by Hopcroft, which works in two steps: Firstly delete unreachable states, then merge equivalent state pairs.

Following this separation in reverse, our approach was to generate the solution DFA first, then create equivalent state pairs and lastly add unreachable states. We devised several sensible input parameters and requirements for each of these stages.

Concerning the generation of solution DFAs we made use of a simple rejection algorithm, that generates test DFAs by randomization or enumeration. Every generated DFA is saved in a database and test DFAs are compared against them, such that new DFAs have a distinct language. On this topic research has already been active, an overview about results there has been made to draw conclusions for this work.

Concerning the extension of solution DFAs towards a task DFA, we found, that we can add states and transitions in an easy manner according to certain rules. These rules were derived from the properties equivalent state pairs and unreachable states have.

\paragraph*{Lookout.}

During our requirements analysis we defined several parameters that have not been or only sparsely further discussed in here. This includes especially boundaries for the number of ingoing transitions to each state and drawing DFAs in a visual comprehensible manner. Connected to the latter is the question, whether a good procedure exists, that outputs a visual representation of a DFA via LaTeX-code, such that hand-made adjustments might be done afterwards. One could also think of making more parameters ranged, such that per instance a minimum and maximum number of states could be specified as input.

Regarding the planarity test as it is used now, one might ask whether there is a more efficient planarity test that is tailored to DFAs. Moreover it could be worth investigating whether informations generated during the planarity test can be used for drawing the DFA.

Our summary on research on DFA generation indicated that efficient - randomized and enumerating - methods to generate DFAs have already been found, whereas the resulting DFAs where even accessible. An improved version of the associated implementation could implement some of these methods or make use of existing implementations. We shall cite in this regard the enumeration method of Almeida et. al.~\cite{AAA09} which uses a similar string representation of DFAs to iterate through all DFAs. Carayol and Nicaud~\cite{CN12} presented a randomization method that is deemed easy to implement.

	
	\appendix
	% List of figures, theorems/lemmas/etc., bibliography, index
	
	\nocite{*}
	\bibliographystyle{abbrv}
	\bibliography{thesis}
	
	\chapter{Erklärung}
	
	Hiermit versichere ich, Gregor Hans Christian Sönnichsen, dass ich die vorliegende Arbeit selbständig verfasst habe, keine anderen als die von mir angegebenen Quellen und Hilfsmittel benutzt habe und die Arbeit nicht  bereits zur Erlangung eines akademischen Grades eingereicht habe. \\
	
	\noindent Bayreuth, den 8.\ Februar 2020. \\\\\\\\\\\\\\
	\noindent Gregor Hans Christian Sönnichsen

\end{document}          
