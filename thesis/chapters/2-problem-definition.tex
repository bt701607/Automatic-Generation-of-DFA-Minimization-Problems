% !TeX spellcheck = en_US

\chapter{Problem definition and approach} \label{ch:2}

In this chapter we will set foundations, investigate sensible parameters and requirements for a minimization task generator and deduce our general approach to build such a program.

\section{Preliminaries}

We start with defining preliminary theoretical foundations. By $[[n]]$ we will denote the set of integers $\{0,\ldots,n-1\}$.

\subsection{Deterministic Finite Automatons}

A 5-tuple $A = (Q, \Sigma, \delta, s, F)$ with $Q$ being a finite set of \emph{states}, $\Sigma$ a finite set of \emph{alphabet symbols}, $\delta \colon\ Q \times \Sigma \to Q$ a \emph{transition function}, $s \in Q$ a \emph{start state} and $F \subseteq Q$ \emph{final states} is called \emph{deterministic finite automaton} (DFA)~\cite[p. 46]{HMU01}. From now on $\A$ shall denote the set of all DFAs.

We say $\delta(q,\sigma) = p$ is a transition from $q$ to $p$ using symbol $\sigma$. We define the \emph{extended transition function} $\delta^* : Q \times \Sigma^* \to Q$ of a DFA $A = (Q, \Sigma, \delta, s, F)$ as:
\begin{itemize}
	\item $\delta^*(q,\varepsilon) = q$
	\item $\delta^*(q,w\sigma) = \delta(\delta^*(q,w),\sigma)$ for all $q \in Q$, $w \in \Sigma^*$, $\sigma \in \Sigma$
\end{itemize}
Then, the \emph{language} of $A$ is defined as $L(A) = \{\ w\ |\ \delta^*(w) \in F\ \}$~\cite[pp. 49-50. 52]{HMU01}.

Given a state $q \in Q$. With $d^-(q)$ we denote the set of all \emph{ingoing} transitions $\delta(q', \sigma) = q$ of $q$. With $d^+(q)$ we denote the set of all \emph{outgoing} transitions $\delta(q, \sigma) = q'$ of $q$~\cite[pp. 2-3]{CP05}.

\begin{definition}[(Un-)Reachable State]\label{ch:1:unreachable-states}
	We say a state $q$ is \emph{(un-)reachable} in a DFA $A$, iff there is (no) a word $w \in \Sigma^*$ such that $\delta^*(s, w) = q$.
\end{definition}
\noindent If all states of a DFA $A$ are reachable, then we call $A$ \emph{accessible}~\cite[p. 2]{CP05}.

A DFA is called \emph{complete} iff for all states, every symbol of the alphabet is used on an outgoing transition: $\forall q\in Q\colon \forall\sigma\in\Sigma\colon \exists p\in Q\colon \delta(q,\sigma) = p$. Note, that every incomplete DFA can be converted to a complete one by adding a so called \emph{dead state}~\cite[p. 67]{HMU01}. The resulting automaton has the same language.

\begin{figure}[H]
	\begin{subfigure}{.5\textwidth}\centering\resizebox{1.\linewidth}{!}{
		\begin{tikzpicture}[initial text={},scale=1., every node/.style={transform shape}]
		\tikzstyle{every state}=[minimum size=5mm, inner sep=0pt]
		
		\node[initial, state]  (0) at (0, 0) {$0$};
		\node[accepting,state] (1) at (2, 1.5) {$1$};
		\node[state]           (2) at (2,-1.5) {$2$};
		\node[state]           (3) at (4, 0) {$3$};
		\node[state]           (4) at (4,-1.5) {$4$};
		
		\path[->]
		(0) edge 			  node [above left=-0.15cm]  {$a$}   (1)
		(1) edge [loop above] node [above=-0.07cm]		 {$a$}   (1)
		(1) edge 			  node [left=-0.07cm]        {$b$}   (2)
		(0) edge 			  node [below left=-0.15cm]  {$b$}   (2)
		(2) edge [loop below] node [below=-0.07cm]       {$a,b$} (2)
		(3) edge 			  node [above right=-0.15cm] {$a$}   (1)
		(4) edge [loop below] node [below=-0.07cm]	     {$a$}   (4)
		;
		\end{tikzpicture}
	}\end{subfigure}
	\hfill
	\begin{subfigure}{.4\textwidth}
		\begin{align*}
			&A = (Q, \Sigma, \delta, s ,F) \\
			&\\
			&Q = [[5]] = \{\ 0,1,2,3,4\ \} \\
			&\Sigma = \{\ a,b\ \} \\
			&\delta = \{\ ((0,a),1), ((0,b),1), \ldots\ \} \\
			&s = 0 \\
			&F = \{\ 1\ \}
		\end{align*}
	\end{subfigure}
	\caption{An example DFA. The states $3$ and $4$ are unreachable. This DFA is not complete since the transitions $\delta(2,b)$ and $\delta(3,b)$ are not defined.}
	\label{fig:dfa}
\end{figure}

\begin{definition}[Minimal DFA]
	We call a DFA $A$ \emph{minimal}, if there exists no other DFA with the same language using less states.
\end{definition}
\noindent With $\Amin$ we shall denote the set of all minimal DFAs.

\begin{definition}[Equivalent and Distinguishable State Pairs]\cite[p. 154]{HMU01}
	A state pair $q_1, q_2 \in Q$ of a DFA $A = (Q, \Sigma, \delta, s, F)$ is called \emph{equivalent}, iff $\sim_A(q_1, q_2)$ is true, where
	\begin{displaymath}
	q_1\ \sim_A\ q_2 \Leftrightarrow_{def}\ \forall z \in \Sigma^* \colon\ (\delta^*(q_1, z) \in F \Leftrightarrow \delta^*(q_2, z) \in F)
	\end{displaymath}
	If $q_0 \not\sim_A q_1$, then $q_0$ and $q_1$ are called a \emph{distinguishable} state pair. The relation $\sim_A$ is an equivalence relation
\end{definition}

\subsection{Isomorphy of DFAs}\label{ch:1:sec:isom}

Given two DFAs $A_1 = (Q_1, \Sigma_1, \delta_1, s_1, F_1)$ and $A_2 = (Q_2, \Sigma_2, \delta_2, s_2, F_2)$. We say $A_1$ and $A_2$ are \emph{isomorph}, iff:
\begin{itemize}
	\item $|Q_1| = |Q_2|$, $\Sigma_1 = \Sigma_2$ and
	\item there exists a bijection $\pi\colon Q_1 \to Q_2$ such that:
	
	$\pi(s_1) = s_2$
	
	$\forall q\in Q_1\colon (q\in F_1 \Longleftrightarrow \pi(q)\in F_2)$
	
	$\forall q\in Q_1\colon \forall\sigma\in\Sigma_1\colon \pi(\delta_1(q,\sigma))=\delta_2(\pi(q),\sigma))$
\end{itemize}
In~\cite[p. 45]{Sch01} we can find the following statement:
\begin{theorem}\label{ch:1:thm:uniq-ism}
	Every minimal DFA is unique (has a unique language) except for isomorphy.
\end{theorem}
\noindent We describe a simple isomorphism test for DFAs in the appendix~\ref{ch:app:ism-test}.

\subsection{The minimization algorithm}

This minimization algorithm \MinAlg\ works in four major steps, essentially removing states in such a way, that no unreachable states and no equivalent state pairs are left.
\begin{enumerate}
	\item Compute all unreachable states via breadth-first search.
	
	\vspace{0.2cm}
	\begin{algorithmic}[1]
		\Function{\CompUnr}{$A$}
			\State $U \gets Q \setminus \{s\}$	\Comment{undiscovered states}
			\State $O \gets \{s\}$				\Comment{observed states}
			\State $D \gets \{\}$				\Comment{discovered states}
			\While {$|O| > 0$}
				\State $N \gets \{\ p\ | \ \exists q \in O\ \sigma \in \Sigma \colon\ \delta(q, \sigma) = p\ \land\ p \notin O \cup D\ \}$
				\State $U \gets U \setminus N$
				\State $D \gets D \cup O$
				\State $O \gets N$
			\EndWhile
			\State \Return $U$
		\EndFunction
	\end{algorithmic}

	\item Remove all unreachable states and their transitions.
	
	\vspace{0.2cm}
	\begin{algorithmic}[1]
		\Function{\RemUnr}{$A, U$}
            \State $\delta' \gets \delta \setminus \{\ ((q,\sigma),p)\in\delta\ |\ q\in U\ \lor\ p\in U\ \}$
			\State \Return $(Q \setminus U, \Sigma, \delta', s, F \setminus U)$
		\EndFunction
	\end{algorithmic}

	\item Compute all equivalent state pairs ($\sim_A$). Inspired by Schöning~\cite[p. 46]{Sch01} and Martens and Schwentick~\cite[p. 17]{MS18}.
	\vspace{0.2cm}
	\begin{algorithmic}[1]
		\Function{\CompDist}{$A$} \label{ch:1:minmark}
		\State $M \gets \{ (p,q), (q,p)\ |\ p \in F, q \notin F \}$
		\Do
			\State $M' \gets \{ (p,q)\ |\ (p,q) \notin M \land \exists \sigma \in \Sigma \colon (\delta(p,\sigma), \delta(q,\sigma)) \in M \}$
			\State $M \gets M \cup M'$
		\doWhile {$M' \neq \emptyset$}
		\State \Return $Q^2 \setminus M$
		\EndFunction
	\end{algorithmic}
	Note that \CompDist\ requires its input automaton to be complete. \gregor{Why?}

	\item Merge all equivalent state pairs, which are exactly those in $\sim_A$. Inspired by Högberg~\cite[p. 10]{HL20}.
	
	\vspace{0.2cm}
	\begin{algorithmic}[1] \label{ch:1:minmerge}
		\Function{\RemEq}{$A$, $\sim_A$}
            \State $Q_E \gets \emptyset$
            \State $\delta_E \gets \emptyset$
            \State $F_E \gets \emptyset$
            \For {$q$ \textbf{in} $Q$}
                \State Add $[q]$ to $Q_E$ \Comment{$[\cdot]_{\sim_A}$ shall be abbreviated $[\cdot]$}
                \For {$\sigma$ \textbf{in} $\Sigma$}
                    \State $\delta_E([q], \sigma) = [\delta(q, \sigma)]$
                \EndFor
                \If {$q \in F$}
                    \State Add $[q]$ to $F_E$
                \EndIf
            \EndFor
			\State \Return $(Q_E, \Sigma, \delta_E, [s], F_E)$
		\EndFunction
	\end{algorithmic}
	Note that \RemEq\ creates complete automatons.
\end{enumerate}

\vspace{0.2cm}
\begin{algorithmic}[1] \label{ch:1:minalg}
    \Function{\MinAlg}{$A$}
    \State $A' \gets \RemUnr(A, \CompUnr(A))$
    \State \Return $\RemEq(A', \CompDist(A'))$
    \EndFunction
\end{algorithmic}
\vspace{0.2cm}
\noindent This DFA minimization algorithm has been found by Hopcroft~\cite{Hop71}.

\begin{theorem}\label{ch:1:min-alg-correct}\textnormal{\cite[pp. 162-164]{HMU01}}
	\MinAlg\ computes a minimal DFA to its input DFA.
\end{theorem}

\noindent When looking at \CompDist, one notes, that it computes distinct subsets of $Q \times Q$ on the way. Indeed, one could write the algorithm in such a way, that these subsets are explicitly computed in form of a function $m\colon\mathbb{N}\to\mathcal{P}(Q\times Q)$:
\vspace{0.2cm}
\begin{algorithmic}[1] \label{ch:1:m-minmark}
	\Function{$m$-\CompDist}{$A$}
	\State $i \gets 0$
	\State $m(0) \gets \{ (p,q), (q,p)\ |\ p \in F, q \notin F \}$
	\Do
		\State $i \gets i + 1$
		\State $m(i) \gets \{ (p,q), (q,p)\ |\ (p,q) \notin \bigcup{m(\cdot)} \land \exists \sigma \in \Sigma \colon (\delta(p,\sigma), \delta(q,\sigma)) \in m(i-1) \}$
	\doWhile {$m(i) \neq \emptyset$}
	\State \Return $\bigcup{m(\cdot)}$
	\EndFunction
\end{algorithmic}
\vspace{0.2cm}
Using this redefinition, we can easier refer to the state pairs marked in a certain iteration. We will use both variants in exchange.
\begin{definition}
	We denote the number of iterations done by \CompDist\ on an DFA $A$ as $\mmD(A)$.
\end{definition}

%\subsection{Another equivalence automaton}
%
%Consider step 3 and 4 of \MinAlg. A view on these algorithms reveal, that they are essentially collapsing each $\sim$-equivalence class of a DFA $A$ into one state.
%
%\begin{definition}[$\sim_A$-equivalence automaton] \label{ch:1:sim-eq-dfa}
%    We will call the automaton $E_A = (Q_E, \Sigma_E, \delta_E, s_E, F_E)$ created by \CompDist\ and \RemEq($A$)\ the \emph{$\sim_A$-equivalence automaton} of $A$. It has the following properties:
%    \begin{itemize}
%        \item $Q_E = \{\ [q]\ |\ q \in Q\ \}$
%        \item $\Sigma_E = \Sigma$
%        \item $\delta_E([q], \sigma) = [\delta(q, \sigma)], \forall q \in Q, \sigma \in \Sigma$
%        \item $s_E = [s]_\sim$
%        \item $F_E = \{\ [q]\ |\ q \in Q\ \}$
%    \end{itemize}
%    Note, that we know by theorem~\ref{ch:1:min-alg-correct} that $E_A$ is guaranteed to be a minimal automaton to $A$, if $A$ has no unreachable states.
%\end{definition}

\section{Requirements analysis}

Now that we have introduced all necessary basic definitions, we shall do a short analysis of an example DFA minimization task and its sample solution, as it could have been given to students in an introductory course to automata theory.

\gregor{search for typical task in standard text books}

\begin{figure}[ht]
	{\raggedright\itshape \underline{Task:} Consider the below shown deterministic finite automaton $A$:}
	\begin{center}\resizebox{.75\linewidth}{!}{
		\begin{tikzpicture}[initial text={},scale=1., every node/.style={transform shape}]
		\tikzstyle{every state}=[minimum size=5mm, inner sep=0pt]
		
		\node[initial, state]  (0) at (0, 0)    {$0$};
		\node[state] 		   (1) at (6, 0)    {$1$};
		\node[state,accepting] (2) at (4,0)     {$2$};
		\node[state]           (3) at (2, -1.5) {$3$};
		\node[state,accepting] (4) at (4,-3)    {$4$};
		\node[state]           (5) at (4,-1.5)  {$5$};
		\node[state]           (6) at (2,0)     {$6$};
		
		\path[->]
		(0) edge node [above=-0.07cm]  {$a$}   (6)
		(0) edge [bend left] node [above=-0.07cm]  {$b$}   (2)
		
		(1) edge node [above=-0.07cm]  {$a$}   (2)
		(1) edge node [below right=-0.15cm]  {$b$}   (4)
		
		(2) edge [bend left] node [above=-0.07cm]  {$a$}   (1)
		(2) edge node [below right=-0.15cm]  {$b$}   (3)
		
		(3) edge node [left=-0.07cm]  {$a$}   (6)
		(3) edge node [above right=-0.15cm]  {$b$}   (4)
		
		(4) edge [bend right] node [below right=-0.15cm]  {$a$}   (1)
		(4) edge [bend left] node [below left=-0.15cm]  {$b$}   (0)
		
		(5) edge node [below right=-0.15cm]  {$a$}   (1)
		(5) edge node [left=-0.07cm]  {$b$}   (4)
		
		(6) edge [bend right] node [below=-0.07cm]  {$a$}   (1)
		(6) edge node [above=-0.07cm]  {$b$}   (2)
		;
		\end{tikzpicture}
	}\end{center}
	\itshape Apply the minimization algorithm and illustrate for each state pair of $A$ during which \CompDist-iteration it was marked. Draw the resulting automaton.
	\caption{An example DFA minimization task.}
	\label{fig:dfa_ex_task}
\end{figure}

\begin{figure}[ht]
	{\raggedright\itshape \underline{Solution:}\newline
		Step 1: Detect and eliminate unreachable states.
		\begin{tabbing}
			\qquad\textnormal{State $6$ is unreachable.}
		\end{tabbing}
		Step 2: Apply \CompDist\ to $A$ and merge equivalent state pairs:\par
	}
	\begin{subfigure}{.4\textwidth}
		\vspace{0.5cm}
		\qquad
		\begin{tabular}{c|c|c|c|c|c|c}
			  & 0  & 1  & 2  & 3  & 4  & 6  \\\hline
			0 & \x & 1  & 0  &    & 0  & 2  \\\hline
			1 & \x & \x & 0  & 1  & 0  & 1  \\\hline
			2 & \x & \x & \x & 0  &    & 0  \\\hline
			3 & \x & \x & \x & \x & 0  & 2  \\\hline
			4 & \x & \x & \x & \x & \x & 0  \\\hline
			6 & \x & \x & \x & \x & \x & \x \\
		\end{tabular}
	\end{subfigure}
	\begin{subfigure}{.5\textwidth}\centering\resizebox{1.\linewidth}{!}{
		\begin{tikzpicture}[initial text={},scale=1., every node/.style={transform shape}]
		\tikzstyle{every state}=[minimum size=5mm, inner sep=0pt]
		
		\node[initial, state]  (03) at (0, 0)   {$03$};
		\node[state] 		   (1) at (6, 0)   {$1$};
		\node[state,accepting] (24) at (4,0)    {$24$};
		\node[state]           (6) at (2,0)    {$6$};
		
		\path[->]
		(03) edge node [above=-0.07cm]  {$a$}   (6)
		(03) edge [bend left] node [above=-0.07cm]  {$b$}   (24)
		
		(1) edge node [above=-0.13cm]  {$a,b$}   (24)
		
		(24) edge [bend left=50] node [above=-0.07cm]  {$a$}   (1)
		(24) edge [bend right=50] node [above=-0.07cm]  {$b$}   (03)
		
		(6) edge [bend right] node [below=-0.07cm]  {$a$}   (1)
		(6) edge node [above=-0.07cm]  {$b$}   (24)
		;
		\end{tikzpicture}
	}\end{subfigure}
	\caption{Solution to the DFA minimization task in fig.~\ref{fig:dfa_ex_task}.}
	\label{fig:dfa_ex_sol}
\end{figure}

\noindent Figures~\ref{fig:dfa_ex_task} and~\ref{fig:dfa_ex_sol} show such a task and solution. The students are confronted with a \emph{task DFA} $A_{task}$. Firstly, unreachable states have to be eliminated, we then gain $A_{re}$. Secondly equivalent state pairs of $A_{re}$ are merged such that the minimal \emph{solution DFA} $A_{sol}$ is found. The table $T$ displayed in figure~\ref{fig:dfa_ex_sol} is nothing else but a visualization of the function $m$ of $m$-\CompDist, whereas $T(q_0, q_1) = i \Leftrightarrow (q_0, q_1) \in m(i)$.

We do some rather formal statements and requirements. Firstly, we can state that
\begin{itemize}
	\item $A_{re} = \RemUnr(A_{task}, \CompUnr(A_{task}))$ and
	\item $A_{sol} = \RemEq(A_{re}, \CompDist(A_{re}))$
\end{itemize}
Therefore $A_{sol}$ is minimal regarding $A_{re}$ and $A_{task}$. Secondly the languages of $A_{task}, A_{re}$ and $A_{sol}$ are equal. We know that \CompDist\ requires $A_{re}$ to be complete and that \RemEq\ creates complete DFAs, so $A_{sol}$ is complete too. Furthermore we know that every state of $A_{re}$ is reachable since it is the output of \RemUnr.

\subsection{Difficulty adjustment possibilities and sensible requirements}\label{ch:1:requirements-analysis}

Concerning the execution of \MinAlg\ we find that its difficulty can be classified through various classification numbers. Furthermore we can note some sensible requirements.

\paragraph*{\CompDist-depth ($\mmD(A_{task})$).}

Consider the computation of the sets $m(i)$ in \CompDist. Determining $m(0)$ is quite straightforward, because it consists simply of tests whether two states are in $F \times Q \setminus F$ (see~\ref{ch:1:m-minmark}, line 3). Determining $m(1)$ is less easy: The rule for determining all $m(i), i > 0$ is different to that for $m(0)$ and more complicated (see~\ref{ch:1:m-minmark}, line 6). Determining $m(2)$ requires the same rule. It shows nonetheless a students understanding of the terminating behavior of \CompDist: It does not stop after computing $m(1)$, but only when no more distinguishable state pairs were found. Concerning the sets $m(i), i > 2$ however no additional understanding can be shown.

It would therefore be sensible if $\mmD(A_{task})$ could be adjusted for example by parameters $m_{min}, m_{max}$ which give lower and upper bounds for that value.

\paragraph*{Number of states \texorpdfstring{($\nSO, \nEQ, \nUN$)}{}.}

To control the number of states in $A_{task}, A_{re}$ and $A_{sol}$, we will introduce three parameters: $\nSO, \nEQ, \nUN \in \mathbb{N}$. These parameters get their meaning by the following equations:
\begin{align*}
    |Q_{sol}| &= \nSO \\
    |Q_{re}| &= \nSO + \nEQ \\
    |Q_{task}| &= \nSO + \nEQ + \nUN
\end{align*}
It is sensible to have $\nUN > 1, \nEQ > 1$, such that \RemUnr\ and \RemEq\ will not be skipped. To not make the task trivial, $\nSO > 2$ is sensible. An exercise instructor will find it useful, to control exactly how big $\nUN$, $\nEQ$ and $\nSO$ are: The higher $\nUN, \nEQ$, the more states have to be eliminated and merged. The higher $\nSO + \nEQ$, the more state pairs have to be checked during \CompDist.

\paragraph*{Alphabet size \texorpdfstring{($\kAL$)}{}.}

The more symbols the alphabet of $A_{task}, A_{re}$ and $A_{sol}$ has (note how \MinAlg\ does not change the alphabet), the more transitions have to be followed when checking whether $(\delta(q,\sigma),\delta(p,\sigma))\in m(i-1)$ is true for each state pair $p,q$.

\paragraph*{Number of final states \texorpdfstring{($\nF$)}{}.}

Since most DFAs in teaching have about 1 to 3 final states, so being able to set a number of final states, that is familiar to students, might be a reason.

\paragraph*{Uniqueness of solution DFA.}

For example for an exam it would be sensible to be able to generate a task, so $A_{sol}$ and $A_{task}$, that has not been generated before. We will use the criterion of isomorphy to distinguish a possible solution DFA from all previously generated ones.

This is sensible since isomorphy distinguishes two minimal DFAs, if and only if they have different languages (see theorem~\ref{ch:1:thm:uniq-ism}).

Note that, if $A_{sol}$ is indeed \emph{new} in that sense, then $A_{task}$ will automatically have a unique language too, since $A_{sol}$ and $A_{task}$ have the same language and this language was then never used before in this context.

\paragraph*{Completeness of task DFA.}

Even though \CompUnr\ and \RemUnr\ do not require their input DFA $A_{task}$ to be complete, it is sensible to build it that way. The implications of the completeness-property are - in comparison to the other concepts involved here - rather subtle. This is especially due to its purely representational nature, a DFA has the same language and $\mmD$-value, whether it is represented in its complete form or not. Nonetheless we shall introduce a parameter $c$, that determines if there exist unreachable states, that make $A_{task}$ incomplete. Thus an exercise lecturer could showcase this matter on a DFA and generate according exercises.

\paragraph*{Planar drawing of task DFA.}

A graph $G$ is \emph{planar} if it can be represented by a drawing in the plane such that its edges do not cross. Such a drawing is then called \emph{planar drawing} of $G$. A visual aid for students would be given, if the task DFA were planar and presented as a planar drawing. In this work libraries and parameters $p_1, p_2 \in \{0,1\}$ (toggling planarity of $A_{sol}, A_{task}$) will be used to allow the option of planarity, but neither ensuring planarity nor planar drawing will be investigated further theoretically.

\paragraph*{Maximum degree of any state in task DFA.}

The \emph{degree} $deg(q)$ of a state $q \in Q$ in a DFA $A$ is defined as $deg(q) = |d^-(q)| + |d^+(q)|$, so the total number of transitions in which $q$ participates. By capping the maximum degree for all states, the graphical representation of the DFA would be more clear. In this work the inclusion of a maximum degree parameter is omitted.

%Note that $deg(q) \geq |\Sigma|$ for any complete DFA, since states of complete DFAs have to use all alphabet symbols on outgoing transitions.

\section{Approach and general algorithm}

In this work we will first build the solution DFA (step 1), and - based on that - the task DFA by creating equivalent states and adding unreachable states (step 2). Both steps will fulfill all criteria chosen above and are covered in depth in chapter~\ref{ch:2} respectively chapter~\ref{ch:3}.

We will see that $\mmD$ and $L$ of both DFAs will be set when building $A_{sol}$. We know that creating equivalent states and adding unreachable does not change $L(A_{task})$ in comparison to $A_{sol}$, else \MinAlg\ would not work (a minimal DFA has in particular the same language as the original DFA). However we must ensure, that adding those states does not change $\mmD$. Since unreachable states are eliminated before \CompDist\ is applied, we need only to prove, that creating equivalent states does not change the $\mmD$-value. We will do this during the discussion of step 2, more specifically in section~\ref{ch:3:sec-D-proof}.

At the beginning of chapter~\ref{ch:2} and~\ref{ch:3}, we will provide formal problem definitions for both steps, that specify precisely all requirements. Here we shall content ourselves with the definition of the main algorithm:
\vspace{0.2cm}
\begin{algorithmic}[1]
	\Function{GenerateDFAMinimizationTask}{$\nSO, \kAL, \nF, m_{min}, m_{max}, p_1, p_2, \nEQ,\nUN, c$}
	\State $A_{sol} \gets \textsc{GenerateNewMinimalDFA}(\nSO, \kAL, \nF, m_{min}, m_{max}, p_1)$
	\State $A_{task} \gets \textsc{ExtendMinimalDFA}(A_{sol}, p_2, \nEQ,\nUN, c)$
	\State \Return $A_{sol}, A_{task}$
	\EndFunction
\end{algorithmic}


