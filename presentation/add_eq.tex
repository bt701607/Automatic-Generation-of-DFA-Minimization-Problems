\begin{frame}
	\frametitle{Extending Minimal DFAs}
	\framesubtitle{Adding Equivalent States (1)}
	
	We now want to add states $r_1,\ldots,r_\nEQ$ to the solution DFA, such that every $r_i$ is equivalent to a state $e$ in the solution DFA:
	\[
	\forall i \in [1,\nEQ] \colon\ \exists e \in Q_{sol}\colon\ r_i \sim_A e
	\]
	Whenever we add a state $r_i$, we will first choose an $e\in Q_{sol}$, then we add the transitions of $r_i$.\pause
	\begin{example}\centering
	\begin{tikzpicture}[initial text={}]
	\tikzstyle{every state}=[minimum size=5mm, inner sep=0pt]
	
	\node[state, initial]  (0) at (0, 0)   {$0$};
	
	\node [fit=(0),draw,rounded corners, dashed] {};
	
	\node[state]  (1) at (2, 0)   {$1$};
	\uncover<2>{\node[state] (2) at (2, -1.5) {$2$};}
	\uncover<3-4>{\node[state,fill=SeaGreen] (2) at (2, -1.5) {$e$};}
	\node         (2p) at (2, -2.4) {};
	\uncover<4>{\node[state,fill=SeaGreen] (ri) at (2, -3) {$r_i$};}
	
	
	\uncover<2-3>{\node[fit=(1)(2)(2p),draw,rounded corners,dashed] {};}
	\uncover<4>{\node[fit=(1)(2)(2p)(ri),draw,rounded corners,dashed] {};}
	
	\node[state,accepting]  (3) at (4, 0)   {$3$};
	\node[state,accepting]  (4) at (4, -1.5)   {$4$};
	
	\node [fit=(3)(4),draw,rounded corners, dashed] {};
	
	\path[->]
	([yshift=0.05cm]0.east) edge node [above] {$a$} ([yshift=0.05cm]1.west)
	([yshift=-0.05cm]0.east) edge node [below] {$b$} ([yshift=-0.05cm]1.west)
	
	(1) edge node [above] {$a$} (3)
	(1) edge node [left] {$b$} (2)
	
	(2) edge node [above] {$a$} (4)
	(2) edge [loop below] node [below] {$b$} (2)
	;
	\end{tikzpicture}
	\end{example}
\note<1>{Sei nE die gewünschte Anzahl an paarweise versch. äqu. Zustandspaaren.\\-\\}
\note<1>{In diesem Verfahren werden die rI einzeln nacheinander hinzugefügt}
\note<2-4>{im Bsp könnten wir zunächst Zst. 2 als orig. aussuchen, und dann rI gedanklich zu dessen Äqu. hinzufügen\\-\\}
\note<2-4>{dann würden wir jetzt die Tr. hinzufügen, dabei beginnen wir mit den ausgehenden Tr.}
\end{frame}

\begin{frame}
	\frametitle{Extending Minimal DFAs}
	\framesubtitle{Adding Equivalent States (2) - Outgoing Transitions}
	
	Observation:
	\[r_i \sim_A e\ \Longrightarrow\ \forall \sigma \in \Sigma \colon [\delta(r_i, \sigma)]_{\sim_A} = [\delta(e, \sigma)]_{\sim_A}\]
	
	Consequently:
	\vspace{0.3cm}
	\begin{itemize}
		\item[R1:] For each symbol $\sigma \in \Sigma$ choose exactly one state $q\in[\delta(e, \sigma)]_{\sim_A}$ and set $\delta(r_i, \sigma) = q$.
	\end{itemize}
	\vspace{0.3cm}\pause

	\begin{example}\centering
	\begin{tikzpicture}[initial text={}]
	\tikzstyle{every state}=[minimum size=5mm, inner sep=0pt]
	
	\node[state, initial]  (0) at (0, 0)   {$0$};
	
	\node [fit=(0),draw,rounded corners, dashed] {};
	
	\node[state]  (1) at (2, 0)   {$1$};
	\node         (2p) at (2, -2.4) {};
	\node[state] (2) at (2, -1.5) {$e$};
	\node[state] (ri) at (2, -3) {$r_i$};
	
	\node[state,accepting]  (3) at (4, 0)   {$3$};
	\node[state,accepting]  (4) at (4, -1.5)   {$4$};
	
	\path[->]
	([yshift=0.05cm]0.east) edge node [above] {$a$} ([yshift=0.05cm]1.west)
	([yshift=-0.05cm]0.east) edge node [below] {$b$} ([yshift=-0.05cm]1.west)
	
	(1) edge node [above] {$a$} (3)
	(1) edge node [left] {$b$} (2)
	;
	
	\uncover<1>{
		\node [fit=(3)(4),draw,rounded corners, dashed] {};
		\node[fit=(1)(2)(2p)(ri),draw,rounded corners,dashed] {};
		\path[->]
		(2) edge node [above] {$a$} (4)
		(2) edge [loop below] node [below] {$b$} (2)
		;
	}
	
	\uncover<2>{
		\node [fit=(3)(4),draw,rounded corners, dashed, color=Blue, line width=1.5pt] {};
		\node[fit=(1)(2)(2p)(ri),draw,rounded corners,dashed] {};
		\path[->]
		(2) edge[color=Blue,line width=1.5pt] node [above] {$a$} (4)
		(2) edge [loop below] node [below] {$b$} (2)
		;
	}
	
	\uncover<3>{
		\node [fit=(3)(4),draw,rounded corners, dashed, color=Blue, line width=1.5pt] {};
		\node[fit=(1)(2)(2p)(ri),draw,rounded corners,dashed] {};
		\path[->]
		(2) edge[color=Blue,line width=1.5pt] node [above] {$a$} (4)
		(2) edge [loop below] node [below] {$b$} (2)
		
		(ri) edge[color=Red,line width=1.5pt] node[below right] {$a$} (4)
		;
	}

	\uncover<4>{
		\node [fit=(3)(4),draw,rounded corners, dashed] {};
		\node[fit=(1)(2)(2p)(ri),draw,rounded corners,dashed, color=Blue, line width=1.5pt] {};
		\path[->]
		(2) edge node [above] {$a$} (4)
		(2) edge[color=Blue,line width=1.5pt] [loop below] node [below] {$b$} (2)
		
		(ri) edge node[below right] {$a$} (4)
		;
	}

	\uncover<5>{
		\node [fit=(3)(4),draw,rounded corners, dashed] {};
		\node[fit=(1)(2)(2p)(ri),draw,rounded corners,dashed, color=Blue, line width=1.5pt] {};
		\path[->]
		(2) edge node [above] {$a$} (4)
		(2) edge[color=Blue,line width=1.5pt] [loop below] node [below] {$b$} (2)
		
		(ri) edge[bend left=1.5cm, color=Red,line width=1.5pt] node[left] {$b$} (1)
		(ri) edge node[below right] {$a$} (4)
		;
	}
	\end{tikzpicture}
	\end{example}
	\note{..dann sind folglich auch diejenigen Zust. äqu., wenn wir ein bel. Symbol von rI und e aus lesen\\-\\}
	\note{und setze diesen Zustand als Endzustand für delta(rI, sigma)\\-\\}
	\note{So konnten wir mit dieser Regel erreichen, dass jede Transition von rI in dieselbe Äqu.kl. führt, die wir err. würden, wenn wir der korresp. Tr. von e folgten}
\end{frame}

\begin{frame}
	\frametitle{Extending Minimal DFAs}
	\framesubtitle{Adding Equivalent States (3) - Ingoing Transitions}
	
	First observation: Since $r_i$ must be reachable, we require $|d^-(r_i)| >= 1$.
	
	\vspace{0.2cm}
	Let $q$ be a state s.t.\ $\delta(q,\sigma)=p$ and we want $\delta(q,\sigma)=r_i$.\pause
	
	\vspace{0.2cm}
	\begin{itemize}
		\item[] $q$ must remain in its equivalence class
		
		\item[$\Rightarrow$] $p$ must be in $[r_i]_{\sim_A} = [e]_{\sim_A}$
		
		\item[$\Rightarrow$] $q$ has to have a transition to some state in $[r_i]_{\sim_A} = [e]_{\sim_A}$\pause
	\end{itemize}
	
	\vspace{0.2cm}
	Furthermore, we see that $p$ must have at least $2$ \emph{ingoing elements}.
	\[
	in(q) = |d^-(q)| + \begin{cases}
	1 & \text{if } s = q\\
	0 & \text{else}
	\end{cases}
	\]\pause
	\begin{itemize}
		\item[R2:] Choose at least one $((q, \sigma), p) \in \delta$ with $[p] = [e]$ and $in(p) \geq 2$. Remove $((q, \sigma), p)$ from $\delta$ and add $((q, \sigma), r_i)$.
	\end{itemize}
	\note<1>{rI kann nicht Startzustand sein, daher nicht in Erreichbarkeits-Bed. aufgenommen\\-\\}
	\note<1>{wir müssen Tr. klauen, weil es keine Tr. mit 'losen' Enden gibt- arbeiten mit kompletten DFAs}
	\note<3>{im Wesentlichen: p muss freie eingehende Tr. haben}
\end{frame}

\begin{frame}
\frametitle{Extending Minimal DFAs}
\framesubtitle{Adding Equivalent States (3) - Ingoing Transitions}

\begin{itemize}
\item[R2:] Choose at least one $((q, \sigma), p) \in \delta$ with $[p] = [e]$ and $in(p) \geq 2$. Remove $((q, \sigma), p)$ from $\delta$ and add $((q, \sigma), r_i)$.
\end{itemize}

	\begin{example}\centering
	\begin{tikzpicture}[initial text={}]
	\tikzstyle{every state}=[minimum size=5mm, inner sep=0pt]
	
	\node[state, initial]  (0) at (0, 0)   {$0$};
	
	\node [fit=(0),draw,rounded corners, dashed] {};
	
	\node[state]  (1) at (2, 0)   {$1$};
	\node         (2p) at (2, -2.4) {};
	\node[state] (2) at (2, -1.5) {$e$};
	\node[state] (ri) at (2, -3) {$r_i$};
	
	\node[fit=(1)(2)(2p)(ri),draw,rounded corners,dashed] {};
	
	\node[state,accepting]  (3) at (4, 0)   {$3$};
	\node[state,accepting]  (4) at (4, -1.5)   {$4$};
	
	\node [fit=(3)(4),draw,rounded corners, dashed] {};
	
	\path[->]
	([yshift=0.05cm]0.east) edge node [above] {$a$} ([yshift=0.05cm]1.west)
	
	(1) edge node [above] {$a$} (3)
	(1) edge node [left] {$b$} (2)
	
	(2) edge node [above] {$a$} (4)
	
	(ri) edge[bend left=1.5cm] node[left] {$b$} (1)
	(ri) edge node[below right] {$a$} (4)
	;
	
	\uncover<1>{
		\path[->]
		([yshift=-0.05cm]0.east) edge[color=Blue, line width=1.5pt] node [below] {$b$} ([yshift=-0.05cm]1.west)
		(2) edge [loop below, color=Blue, line width=1.5pt] node [below] {$b$} (2)
		;
	}
	
	\uncover<2>{
		\path[->]
		([yshift=-0.05cm]0.east) edge[color=Blue, line width=1.5pt] node [below] {$b$} ([yshift=-0.05cm]1.west)
		(2) edge[color=Red, line width=1.5pt] node[left] {$b$} (ri)
		;
	}

	\uncover<3>{
		\path[->]
		(0) edge[bend right=1.5cm, color=Red, line width=1.5pt] node[left] {$b$} (ri)
		(2) edge[color=Red, line width=1.5pt] node[left] {$b$} (ri)
		;
	}
	\end{tikzpicture}
\end{example}

\end{frame}