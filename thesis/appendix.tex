% !TeX spellcheck = en_US

\chapter{An isomorphy test for DFAs}\label{ch:app:ism-test}

Here follows a simple isomorphy test that tries essentially to build a bijection as described in section~\ref{ch:2:sec:isom}.
\vspace{0.2cm}
\begin{algorithmic}[1]
	\Function{AreIsomorphic\ }{$A_1, A_2$}
		\If {$|Q_1| \neq |Q_2|$ \textbf{or} $|F_1| \neq |F_2|$ \textbf{or} $\Sigma_1 \neq \Sigma_2$}
			\State \Return $false$
		\EndIf
		
		\State $\pi(s_1) = s_2$ \Comment{bijection $Q_1 \to Q_2$}
		\State $O \gets \emptyset$ \Comment{observed states}
		\State $V \gets \{s_1\}$ \Comment{visited states}
		\State $q_c \gets s_1$ \Comment{current state}
		
		\While {true}
			\For {$((q_1,\sigma),p_1)$ \textbf{in} $\delta_1$} \Comment{iterate through $d^+(q_c)$}
				\If {$q_1 \neq q_c$} 
					\State \textbf{continue}
				\EndIf
				
				\State
				
				\State $p_2 \gets \delta_2(\pi(q_c), \sigma)$ 
				\State $p1marked \gets (\pi(p_1) \neq \bot)$ \Comment{see if $p_1$, $p_2$ were ``marked'' by $\pi$}
				\State $p2marked \gets (\exists q\colon \pi(q)=p2)$
				
				\State
				
				\If {$p1marked$ \textbf{and} $p2marked$}
					\If {$\pi(p_1) \neq p_2$}
						\State \Return false
					\EndIf
				\ElsIf {$\neg p1marked$ \textbf{and} $\neg p2marked$}
					\State $\pi(p_1) = p_2$
				\If {$p_1 \notin V$}
					\State Add $p_1$ to $O$
				\EndIf
				\Else \Comment{one of $p_1$, $p_2$ was assigned to some state $\neq p_1$ resp.\ $p_2$}
					\State \Return false
				\EndIf
			\EndFor	
			
			\If {$|O| = 0$}
				\State \textbf{break}
			\EndIf
			\State Pick and remove $q_c$ from $O$
			\State Add $q_c$ to $V$
		\EndWhile
		\State\textbf{end}
		
		\For {$q_1$ \textbf{in} $F_1$}
			\If {$\pi(q_1) \notin F_2$}
				\State \Return false
			\EndIf
		\EndFor	
		
		\State \Return true
	\EndFunction
\end{algorithmic}
\vspace{0.2cm}
\newpage
\noindent This algorithm \emph{visits} all states of $A_1$ one by one and tries to build $\pi$ on the way. The currently visited state is denoted $q_c$. In $V \subseteq Q_1$ we save all already visited states. The set $O \subseteq Q_1$ shall contain all \emph{observed} states, meaning those, that we encountered while following a transition, but have not visited yet.

We call states of $A_1$, $A_2$ \emph{marked}, if they have been assigned to another state by $\pi$. So if $\pi(q_1)=q_2$, then $q_1$, $q_2$ are marked. States in $O$ will have the property that they are marked.

Starting with $q_c = s_1$, in every while-iteration all outgoing transitions $\delta_1(q_c, \sigma) = p_1$ of the current state are followed. We then compute $\delta_2(\pi(q_c), \sigma) = p_2$, which is the state in $A_2$ that should correspond to $p_1$ of $A_1$. At this point (line 17) we do a case differentiation:
\begin{itemize}
	\item $p_1$, $p_2$ both marked: We only need to ensure they are assigned to each other.
	\item $p_1$, $p_2$ both not marked: We can assign them to each other and may now add $q_1$ to $O$, since we observed it on an outgoing transition and know it has been marked. We will not add it, if we have visited as $q_c$.
	\item one of $p_1$, $p_2$ marked, one not: In that case one of both states has already been assigned to another state. Thus the construction of a bijection has failed here, since $p_1$, $p_2$ should belong together.
\end{itemize}
When finished with visiting all outgoing transitions of a state $q_c$, we can pick the next state which is added to the visited states.

If all states of $A_1$ have been visited and the bijection thus been fully constructed, we need only to ensure, that the final state sets are equal after a renaming according to $\pi$.