% !TeX spellcheck = en_US

\chapter{Extending Minimal DFAs} \label{ch:4}

We firstly define a formal problem for extending a minimal DFA $A_{sol}$ to a task DFA $A_{task}$ based on our requirements analysis (see sec.~\ref{ch:2:requirements-analysis}):
\begin{definition}[ExtendMinDFA] $ $ \\
	$ $ \vspace{-0.cm} \\
	\noindent $\underline{\emph{Given:}}$
	\vspace{-0.2cm}
	\begin{align*}
	A_{sol} = (Q, \Sigma, \delta, s, F) \in \Amin\ \ \ & \emph{solution DFA} \\
	\nEQ \in \mathbb{N}\ \ \ & \emph{number of states creating equivalent state pairs} \\
	\nUN \in \mathbb{N}\ \ \ & \emph{number of unreachable states} \\
	p \in \{0,1\}\ \ \ & \emph{planarity-bit} \\
	c \in \{0,1\}\ \ \ & \emph{completeness-bit for unreachable states}
	\end{align*}
	\noindent $\underline{\emph{Task:}}$ \emph{Compute, if it exists, a task DFA $A_{task}$ with}
	\begin{itemize}
		\item $Q_{task} = Q_{sol} \cup \{ r_1, \ldots, r_\nEQ, u_1, \ldots, u_\nUN \}$
		\item $r_1, \ldots, r_\nEQ$ \emph{each creating an equivalent state pair\footnotemark}
		\item $u_1, \ldots, u_\nUN$ \emph{unreachable}
		\item $\Sigma_{task} = \Sigma_{sol}$, $s_{task} = s_{sol}$, $F_{task} \subseteq F_{sol}$
		\item $A_{task}$ \emph{being planar iff} $p = 1$
		\item $A_{task}$ \emph{being complete iff} $c = 1$
		\item $A_{sol}$ \emph{being isomorphic to} $\MinAlg(A_{task})$
	\end{itemize}
\end{definition}
\footnotetext{The states $r_1, \ldots, r_\nEQ$ can be seen as being \emph{redundant}.}
\noindent In order to fulfill these requirements we will deduce for equivalent and unreachable states how they may be added by examining their desired properties. If we get to have the choice between several equally good options, we will randomly select one.

We will be guided by the separation of creating equivalent and unreachable states as in Hopcrofts algorithm, thus distinct algorithms will be devised to create each kind of states. Concerning the planarity option, we again use a rejection algorithm with the same test as before (see sec.~\ref{ch:3:sec:planarity}). These decisions lead to the following formulation of \textsc{ExtendMinDFA}:
\newpage
\vspace{0.2cm}
\begin{spacing}{1}
	\begin{algorithmic}[1]
		\Function{ExtendMinDFA}{$A_{sol}, \nEQ, \nUN, p, c$}
		\If{$p = 0$}
			\State $A_{re} \gets$ \textsc{AddUnrStates}($A_{sol}, \nUN, c$)
			\State \Return \textsc{CreateEquivPairs}$(A_{re}, \nEQ$)
		\Else
			\State $A_{task} \gets \bot$
			\While{$A_{task}$ not planar}
				\State $A_{re} \gets$ \textsc{AddUnrStates}($A_{sol}, \nUN, c$)
				\State $A_{task} \gets$ \textsc{CreateEquivPairs}$(A_{re}, \nEQ$)
			\EndWhile
			\State \Return $A_{task}$
		\EndIf
		\EndFunction
	\end{algorithmic}
\end{spacing}
\vspace{0.2cm}
\noindent We will show for the action of adding equivalent states, that this does not influence a DFAs $\mmD$-value --- thus we do not have to care whether we are possibly changing it.

\section{Creating Equivalent State Pairs}

Step 3 and 4 of the minimization algorithm are concerned with detection and elimination of equivalent state pairs. We now want to add states $r_1,\ldots,r_\nEQ$ to a DFA $A_{sol}$, gaining $A_{re}$ with $Q_{re} = Q_{sol} \cup \{r_1,\ldots,r_\nEQ\}$, such that each of the added states is equivalent to a state in $Q_{re}$. Note that, for reasons of clarity, we are going to abbreviate from now on $A_{re} = A$, $Q_{re} = Q$, $\sim_{A_{re}} = \sim_A$ etc.

In our algorithm we will add each $r = r_i,\ i\in[1,\nEQ]$ separately to $A_{sol}$. Consider the properties $r_1,\ldots,r_\nEQ$ must have. Since we start from $A_{sol}$, and add in each step a state, that will be equivalent to a state in the so-far constructed DFA, it follows by transitivity, that each $r$ will be equivalent to an \emph{origin} state $e$ of $A_{sol}$.
\[
	\forall r \in \{r_1,\ldots,r_{\nEQ}\} \colon\ \exists e \in Q_{sol}\colon\ r \sim_A e
\]
In our algorithm, we will first choose a to-be-equivalent-state $e\in Q_{sol}$ for each state $r$ we create, then we add its transitions.

\subsection{Adding Outgoing Transitions}

Regarding the outgoing transitions of any $r$ equivalent to a state $e$, we are directly restricted by their equivalence relation:
\begin{itemize}
	\item[] $r \sim_A e$
	\item[$\Rightarrow$] $\forall z \in \Sigma^* \colon\ (\delta^*(r, z) \in F \Leftrightarrow \delta^*(e, z) \in F)$
	
	\item[$\Rightarrow$] $\forall \sigma \in \Sigma \colon$
	
	\qquad $\delta(r, \sigma) = q_1 \land \delta(e, \sigma) = q_2 \land$
	
	\qquad $\forall z' \in \Sigma^*\colon (\delta^*(q_1, z') \in F \Leftrightarrow \delta^*(q_2, z') \in F)$
	
	\item[$\Rightarrow$] $\forall \sigma \in \Sigma \colon$
	
	\qquad $\delta(r, \sigma) = q_1 \land \delta(e, \sigma) = q_2 \land q_1 \sim_A q_2$
	
	\item[$\Rightarrow$] $\forall \sigma \in \Sigma \colon [\delta(r, \sigma)]_{\sim_A} = [\delta(e, \sigma)]_{\sim_A}$
\end{itemize}
So we see, that the state $\delta(r, \sigma) = q$ must be in the same equivalence class as $\delta(e, \sigma) = p$. We may thus formulate the rule for adding outgoing transitions to a new state quite straightforwardly:
\begin{description}
	\item[R1:] For each symbol $\sigma \in \Sigma$ choose exactly one state ($A$ shall be complete) $q\in[\delta(e, \sigma)]_{\sim_A}$ and set $\delta(r, \sigma) = q$.
\end{description}
Since the solution DFA is complete and since every here added state gets a transition for every alphabet symbol, we know that every $[\delta(e, \sigma)]_{\sim_A} \neq \emptyset$, so the rule is guaranteed to be fulfillable.

Note that this substep does not change the equivalence class of any other now existing state, since $r$ cannot be reached yet (and thus it can not lie on a path from any other state to a final state) - it has no ingoing transitions.

\subsection{Adding Ingoing Transitions}

First of all, we know, that $r$ must be reachable, since we decided that all unreachable states will be added later. So we need to give $r$ at least one ingoing transition. Doing this, we have to ensure, that any state $q$, that gets an outgoing transition to $r$ remains in its equivalence class.

Thus a fitting state $q$ has to have a transition to some state in $[r]_{\sim_A} = [e]_{\sim_A}$ already. So, given a state $q$ with $\delta(q, \sigma) = p$ and $p \in [e]_{\sim_A}$, we can set $\delta(q, \sigma) = r$ and thereby ``steal'' $p$ its ingoing transition.

We see here, that $p$ must have at least one ``free'' transition, else it would become unreachable. A state has a free transition, if it has at least two ingoing transitions or if it is start state and has at least one ingoing transition. With $in(q)$ we denote the \emph{ingoing elements of $q$}:
\[
	in(q) = |d^-(q)| + \begin{cases}
							1 & \text{if } s = q1\\
							0 & \text{else}
						 \end{cases}
\]
Now we can define our rule for adding ingoing transitions:
\begin{description}
	\item[R2:] Choose at least one $((q, \sigma), p) \in \delta$ with $[p] = [e]$ and $in(p) \ge 2$. Remove $((q, \sigma), p)$ from $\delta$ and add $((q, \sigma), r)$.
\end{description}

\subsection{Requirements for choosing origin states}

From rule \textbf{R2} we know that the equivalence class of any origin state $e$ has to contain at least one state with at least two ingoing elements. We establish the following notion to pin down the requirement which all origin states have to fulfill when they are chosen:
\[
duplicatable(q) \Leftrightarrow_{def} (\exists p \in [q]_{\sim_A}\colon in(p)\ge 2)
\]
\begin{observation}
	The number of duplicatable states in any accessible DFA $A$ is
	\begin{description}
		\item$= 0$, if $|\Sigma| \leq 1$ and $|d^-(s)| = 0$: The start state has no free ingoing transition and neither have all other states.
		
		\item$= 1$, if $|\Sigma| \leq 1$ and $|d^-(s)| > 0$: The start state has $|d^-(s)| - 1$ free ingoing transitions, but other states still do not have enough.
		
		\item$> 1$, if $|\Sigma| > 1$, due to the pigeonhole principle: An accessible complete DFA has $|Q||\Sigma|$ transitions which have to be spread across $|Q|$ states.
	\end{description}
\end{observation}
\noindent This observation guarantees, that there exists a duplicatable state, if $k > 1$ or $k = 1$ and $|d^-(s)| > 0$. Note that we do not guarantee, that $\nEQ$ equivalent states can be added.
\begin{lemma}
	Adding the outgoing transitions of $r$ does not change whether its associated state $e$ is duplicatable.
\end{lemma}
\begin{proof}
Case $1$: $e$ is duplicatable. Adding transitions to the DFA will never change that, since $|d^-(e)|$ will stay the same or increase, but never decrease.\par
	
Case $2$: $e$ is not duplicatable. Towards a contradiction: The only outgoing transition from $r$, that could make $e$ duplicatable, would be $\delta(r, \sigma) = e$.
	
Adding this transition by rule \textbf{R1} would require, that there exists a transition $\delta(e, \sigma) \in [e]$ (see \textbf{R1}).
	
By the definition of $\sim_A$, every state in $[e]$ then has to have an outgoing transition to a state in $[e]$.
	
Since $e$ is not duplicatable, all states in $[e]$ have at maximum one ingoing elements. If $s \in [e]$, then there are not enough ingoing transitions to match all these outgoing transitions.

Else the required $|[e]|$ outgoing transitions would form all ingoing transitions of the $|[e]|$ states in $[e]$.
	
But then there would be no ingoing element left that ensures the reachability of $[e]$.
\end{proof}

\subsection{The Algorithm}

We integrate our strategy of creating equivalent state pairs in the following algorithm. It basically consists of a for-loop that adds one state per iteration. In each iteration first a duplicatable state in the so-far built DFA is determined. Then the new state is created and outgoing and ingoing transitions are added.

\vspace{0.2cm}
\begin{spacing}{1}
\begin{algorithmic}[1]
	\Function{CreateEquivPairs}{$A, \nEQ$}
	\If {$k = 0$ \textbf{or} ($k = 1$ \textbf{and} $|d^-(s)| = 0$)}
		\State \Return $\bot$
	\EndIf
    \State $Q \gets Q_{sol}$
    \State $\delta \gets \delta_{sol}$
    \State $F \gets F_{sol}$
    \State $\sim_A \gets Q\times Q$
%	\State $K \gets \{\ \{q\}\ |\ q \in Q\ \}$ \Comment{tracks the equivalence classes of $A$}
%	\State $k(q) = C$ such that $q \in C$ and $C \in K$ \Comment{returns the equivalence class to $q$}
	
	\vspace{0.2cm}
	
    \For {$i$ \textbf{in} $[1,\nEQ]$}
    
    	\vspace{0.2cm}
    
		\For {$q$ \textbf{in} $Q$} \Comment{find a duplicatable state $e$}
			\If {$in(q) \ge 2$}
				\State $e \gets$ random chosen state from $[q]_{\sim_A}$
				\State \textbf{break}
			\EndIf
		\EndFor
		\If {$e = \bot$} \Return $\bot$ \EndIf
		
		\vspace{0.2cm}
		
		\State $r \gets$ unused state label \Comment{create to $e$ equivalent state $r$}
        \State Add $r$ to $Q$
		\State Add $(r,r), (e,r), (r,e)$ to $\sim_A$
		
		\vspace{0.2cm}
		
		\For {$\sigma$ \textbf{in} $\Sigma$} \Comment{\textbf{R1:} add $d^+(r)$}
			\State $\delta(r, \sigma) =$ random chosen state from $[\delta(e, \sigma)]_{\sim_A}$
		\EndFor
		
		\vspace{0.2cm}
		
		\State $P \gets \{\ ((s, \sigma), t) \in \delta\ |\ t \in [e]_{\sim_A},\ in(t) \geq 2\ \}$ \Comment{\textbf{R2:} add $d^-(r)$}
		\State $C \gets$ random nonempty subset of $P$
		\For {$((s, \sigma), t)$ \textbf{in} $C$}
            \State $\delta(s, \sigma) = r$
		\EndFor
		
		\vspace{0.2cm}
		
	\EndFor
    \State \Return $(Q, \Sigma_{sol}, \delta, s_{sol}, F)$
	\EndFunction
\end{algorithmic}
\end{spacing}
\vspace{0.2cm}
\noindent Note that computing an unused state label can be easily done by e.g.\ taking the maximum of all solution DFA states (which are nothing else but numbers) and adding one.



\subsection{Creating Equivalent State Pairs does not change $\mmD$} \label{ch:4:sec-D-proof}

In this section we want to prove that our method of creating state pairs does not affect the number of \CompDist-iterations. Using this information we can be sure that $\mmD(A_{sol}) = \mmD(A_{task})$ and our just explained algorithm does not have to care about possibly changing this value.

To do this proof, we will first introduce two auxiliary definitions and then prove two minor lemmas. As a side effect, Lemma~\ref{ch:4:semantics-of-m(n)} will describe a central property of \CompDist\ and Lemma~\ref{ch:4:semantics-of-D(A)} will show an extended characterization of $\mmD(A)$ compared to its definition (def.~\ref{ch:2:def:D(A)}).

A word $w$ shall be called \emph{finishing word of $q$}, iff $\delta^*(q, w) \in F$.

With $f(q) = \{\ w\ |\ \delta^*(q, w) \in F\ \}$ we denote the set of all finishing words to a state.
\begin{definition} \label{ch:4:def-dist-word}
	We will call a word $w$ \emph{distinguishing word of $p,q$}, iff $d_A(w, p, q)$ is true where
	\begin{align*}
	d_A(w, p, q) \text{ is true} &\Leftrightarrow (\delta^*(p,w) \in F \Leftrightarrow \delta^*(q,w) \notin F) \\
	&\Leftrightarrow (w \in f(p) \Leftrightarrow w \notin f(q))
	\end{align*}
\end{definition}
\noindent This definition and its terminology are in close relation to definition~\ref{ch:2:def:eq-dist-pairs}. The following lemma and its proof are in parts inspired by Martens and Schwentick \cite[ch.\ 4 p.\ 18]{MS18}.

\begin{lemma}\label{ch:4:semantics-of-m(n)}
    In the context of \CompDist\ the following is true: Iff $(p,q)\in m(n)$, the shortest distinguishing word of $p,q$ has length $n$. Formally:
    \begin{align*}
        (p,q) \in m(n) \Longleftrightarrow\ &\exists w\in\Sigma^*\colon (|w| = n\ \land d_A(w, p, q))\\
        \land\ &\nexists v\in\Sigma^*\colon (|v| < n\ \land d_A(v, p, q))
    \end{align*}
\end{lemma}

\begin{proof}
	Per induction on the number of \CompDist-iterations $n$.
	
	\paragraph*{$n = 0$, ``$\Leftrightarrow$''.}
	\begin{align*}
		&(p,q) \in m(0) = \{ (p,q), (q,p)\ |\ p \in F, q \notin F \}\hfill\text{ (see alg.~\ref{ch:2:m-minmark}, line 2))}\\
		\Leftrightarrow\ &\text{one of $p,q$ in $F$, one not}\\
		\Leftrightarrow\ &\text{one of $\delta^*(p, \varepsilon),\delta^*(q, \varepsilon)$ in $F$, one not}\\
		\Leftrightarrow\ &\exists w\in\Sigma^*\colon (|w|=0\land\text{one of $\delta^*(p, w),\delta^*(q, w)$ in $F$, one not})\\
		\Leftrightarrow\ &\exists w\in\Sigma^*\colon (|w| = 0\ \land d_A(w, p, q))\\
		&\text{and there is no shorter such word }\checkmark
	\end{align*}

\paragraph*{$n > 0$, ``$\Rightarrow$''.} 
Then the following holds for some states $p,q$ (see alg.~\ref{ch:2:m-minmark}, line 5):
\begin{equation}\label{ch:4:eq:m(n)}
(p,q) \in \{ (p,q), (q,p)\ |\ (p,q) \notin \bigcup{m(\cdot)} \land \exists \sigma \in \Sigma \colon (\delta(p,\sigma), \delta(q,\sigma)) \in m(n-1) \}
\end{equation}
We will prove: There exists a distinguishing word of length $n-1$ for $p,q$, and there is no shorter distinguishing word for $p,q$.

Looking at eq.~\ref{ch:4:eq:m(n)} we observe, there exists a symbol $\sigma$ such that $(\delta(p,\sigma),\delta(q,\sigma)) \in m(n-1)$. Let $p',q'=\delta(p,\sigma),\delta(q,\sigma)$, so $(p',q')\in m(n-1)$.

Per induction there exists a (shortest) distinguishing word $w'$, $|w'|=n-1$ to $p',q'$. Thus one of $\delta^*(p', w'),\delta^*(q', w')$ is in $F$, one not.

Thus one of $\delta^*(p, \sigma w'),\delta^*(q, \sigma w')$ is in $F$, one not, which makes $\sigma w'$ a distinguishing word of length $n$ for $p,q$.

Since $(p,q)$ is not in any $m(i), i<n$ (recall $(p,q) \notin \bigcup{m(\cdot)}$ of eq.~\ref{ch:4:eq:m(n)}), there is per precondition no shorter distinguishing word for $p,q$, making $\sigma w'$ (a) shortest distinguishing word for $p,q$.\ $\checkmark$ 

\paragraph*{$n > 0$, ``$\Leftarrow$''.} 
Then the following holds for some states $p,q$:
\begin{align*}
&\exists w\in\Sigma^*\colon (|w| = n\ \land d_A(w, p, q))\\
\land\ &\nexists v\in\Sigma^*\colon (|v| < |w|\ \land d_A(v, p, q))
\end{align*}
Since $w$ is non-empty there exists a symbol $\sigma$ such that $w = \sigma w'$. Let $\delta(p,\sigma),\delta(q,\sigma) = p',q'$.

Thus, if one of $\delta^*(p, \sigma w'),\delta^*(q, \sigma w')$ is in $F$ and one not, then the same must hold for $\delta^*(p', w'),\delta^*(q', w')$, so $w'$ is a distinguishing word for $p',q'$.

It is also the shortest one, because, if there existed a shorter word $v'$, $|v'| < |w'|$, then $\sigma v'$ would be a distinguishing word shorter than $w$ for $p,q$ which is contradictory.

Since $w'$ is a shortest distinguishing word for $p',q'$, we may deduce now per induction, that $(p',q')\in m(n-1)$.

The pair $(p,q)$ is not in any $m(i)$, $i<n$, since otherwise per induction the shortest distinguishing word would be shorter than $w$ and thus not $w$. Since $(p',q')\in m(n-1)$ and $\delta(p,\sigma),\delta(q,\sigma) = p',q'$, we can then deduce by the definition of $m$, that $(p,q)\in m(n)$.\ $\checkmark$ 
\end{proof}

\begin{lemma}\label{ch:4:semantics-of-D(A)}
    If \CompDist\ has done $n$ iterations and terminated (so $\mmD(A) = n$), then the longest word $w$, that is a shortest distinguishing word for any state pair, has length $\mmD(A)-1$.
\end{lemma}

\begin{proof}
	Via direct proof. Assume $m$-\CompDist(A) has done $n$ iterations (so $\mmD(A) = n$). We observe, that
	\begin{enumerate}
		\item $\forall i \in [0,n-1]\colon m(i) \neq \emptyset$
		\item $m(n)= \emptyset$
		\item $\forall i > n\colon m(i)= \bot$\ .
	\end{enumerate}
	This follows directly from while loop and its terminating condition of \CompDist\ (alg.~\ref{ch:2:m-minmark}, line 4--7). Given this, we will prove: There exists a shortest distinguishing word of length $n-1$ for some state pair, but a longer such word cannot exist.

	% a possible word per definition of D(A), m(i) and lemma
	
	Following Lemma~\ref{ch:4:semantics-of-m(n)} and the first observation, we can deduce the existence of a shortest distinguishing word $w$ with $|w| = n-1 = \mmD(A)-1$ for some $p,q \in Q$.
	
	% There is no word longer than that
	
	There cannot be any shortest distinguishing word $w'$ with $|w'| = k > n-1$ for any two states $p',q'\in Q$. Following Lemma~\ref{ch:4:semantics-of-m(n)} again, $m(k)$ for some $k > n-1$ would be defined and non-empty, which is contradictory to observations 2 and 3.
\end{proof}

\begin{theorem}\label{ch:4:th-D}
	Given two DFAs $A$, $A'$. If both are accessible and their language is the same $(L(A) = L(A'))$, then \CompDist\ runs with the same number of iterations on them $(\mmD(A) = \mmD(A'))$.
\end{theorem}

\begin{proof}
	Starting with the language-equivalence of $A$ and $A'$ we observe, that the start states of both DFAs have the same finishing words.
	\begin{itemize}
		\item[] $L(A) = L(A')$
		
		\item[$\Rightarrow$] $\{\ w\ |\ \delta^*(s, w) \in F\ \} = \{\ w\ |\ \delta'^*(s', w) \in F'\ \}$
		
		\item[$\Rightarrow$] $\forall w \in \Sigma^*\colon \delta^*(s, w) \in F \Leftrightarrow \delta'^*(s', w) \in F'$
	\end{itemize}
	We extend this to a statement that includes any state visited on the way to $F$ resp.\ $F'$. We can see, that those states reached by the same word in $A$, $A'$ have the same finishing words.
	\begin{itemize}
		\item[] $\forall u \in \Sigma^*\colon \exists q,q' \in Q\colon$
		
		\qquad $\delta^*(s, u) = q \land \delta'^*(s', u) = q' \land$
		
		\qquad $(\forall v \in \Sigma^*\colon (\delta^*(q, v) \in F \Leftrightarrow \delta'^*(q', v) \in F'))$
		
		\item[$\Rightarrow$] $\forall u \in \Sigma^*\colon \exists q,q' \in Q\colon$
		
		\qquad $\delta^*(s, u) = q \land \delta'^*(s', u) = q' \land$
		
		\qquad $f(q) = f(q')$
	\end{itemize}
	Since we are making a statement about all states reached from $s$/$s'$ and since all states in $A$/$A'$ are reachable, we may conclude:
	
	For every state in $A$/$A'$ there exists a state in the other DFA, such that their finishing words are the same.
	\begin{itemize}
		\item [] $\forall q \in Q\colon \exists q' \in Q'\colon f(q) = f(q')$ \hfill $\land$ \hfill $\forall q' \in Q'\colon \exists q \in Q\colon f(q) = f(q')$ \qquad \qquad \qquad \qquad
		
		\item[$\Rightarrow$] $\{\ f(q)\ |\ q \in Q\ \} = \{\ f(q')\ |\ q' \in Q'\ \}$
	\end{itemize}
	Since a distinguishing word is defined as being finishing word for one state and for one not (see def.~\ref{ch:4:def-dist-word}), there cannot be a distinguishing word in one of $A$/$A'$, that is not distinguishing word in the other DFA.
	
	As a consequence both DFAs have the same shortest distinguishing words and thus too the same longest shortest distinguishing word.
	
	If $\mmD(A) \neq \mmD(A')$ then by Lemma~\ref{ch:4:semantics-of-D(A)} one DFA would have a longer longest shortest distinguishing word, which is not true as proven, thus $\mmD(A) = \mmD(A')$ must be true. 
\end{proof}

\begin{corollary}
	Our method of creating equivalent state pairs in a DFA does not change the DFAs $\mmD$-value.
\end{corollary}

\begin{proof}
	Creating equivalent state pairs does not change the language of a DFA --- otherwise the reverse procedure, \MinAlg, could not guarantee the language stays the same. So since the language stays the same when creating those pairs, by Theorem~\ref{ch:4:th-D} the number of \CompDist-iterations will be preserved as well.
\end{proof}

\section{Adding Unreachable States}

Unreachable states are states, which are not reachable from the start state (see def.~\ref{ch:2:unreachable-states}). Consequently there are no restrictions on number and nature of their outgoing transitions. Let us say the added states is $q$.

Concerning $q$'s ingoing transitions, we only have to ensure, that none of them allows reaching $q$ from the start state. This implicates, that only other unreachable states are qualified as start points of those ingoing transitions.

Note that, given $q, \sigma$, we allow only one end point to $\delta(q, \sigma)$. Consequently if we want to add an ingoing transition $\delta(q_u, \sigma) = q$ where $q_u$ is unreachable, then we are stealing the transition from another state $\delta(q_u, \sigma) = q'$. But this is okay, since $q'$ must be an unreachable state too and will stay that way if we remove an ingoing transition.

In this algorithm we assume, that $A$ has no unreachable states.
\vspace{0.2cm}
\begin{algorithmic}[1]
	\Function{AddUnrStates\ }{$A, \nUN, c$}
	\State $U \gets \emptyset$ \Comment{unreachable states in $A$}
	\vspace{0.2cm}
	
	\For {$\nUN$ \textbf{times}}
	\vspace{0.2cm}
	
		\State $q \gets$ unused state label
		\State Add $q$ to $Q$
		\vspace{0.2cm}
		
		\For {$q_u, \sigma$ \textbf{in} random subset of $U\times\Sigma$} \Comment{in. transitions}
			\State $\delta(q_u, \sigma) = q$
		\EndFor
		\vspace{0.2cm}
		
        \State $\Sigma' \gets \textbf{if } c = 1 \textbf{ then } \Sigma \textbf{ else}$ random subset of $\Sigma$ \Comment{out. transitions}
		\State $S \gets$ random chosen sample of $|\Sigma'|$ states from $Q$
		\For {$\sigma$ \textbf{in} $\Sigma'$}
			\State Remove a state $q'$ from $S$
			\State $\delta(q, \sigma) = q'$
		\EndFor
		\vspace{0.2cm}
		
		\State Add $q$ to $U$
	\EndFor
	\vspace{0.2cm}
	
	\State \Return $A$
	\EndFunction
\end{algorithmic}
\vspace{0.2cm}
If completeness is demanded ($c=1$), then we set $\Sigma$ as set of all symbols, for which a state shall gain outgoing transitions. Else we choose a random subset for each state, such that some unreachable states may miss some outgoing transitions.
