% !TeX spellcheck = de_DE

\chapter{Abstract}

% chapter 1

The theory of deterministic finite automata (DFAs) is a classical topic of computer science-related courses. A typical problem to solve for students is: Given a task DFA $A_{task}$, minimize it. Generation of such problems is often done manually by the exercise instructor. This work presents ideas to automize the generation of DFA minimization problems.

% chapter 2

We start in chapter~\ref{ch:2} with introducing some preliminaries, in particular Hopcroft's minimization method. Based on that algorithm, we will deduce requirements, parameters and structure for a generator of DFA minimization problems. The structure will be the following: Firstly generate the solution DFA, then create equivalent state pairs and lastly add unreachable states. This approach is, intuitively spoken, Hopcroft's algorithm in reverse.

% chapter 3

Concerning the generation of solution DFAs (chapter~\ref{ch:3}) we make use of a simple rejection algorithm, that generates test DFAs by randomization or enumeration and rejects them, if they do not match the demanded properties. From results in related research some conclusions will be drawn for this work.

% chapter 4

In chapter~\ref{ch:4} we describe the extension of a solution DFA towards a task DFA. To archive this, we can add states and transitions in an easy manner according to certain rules, which are derived from the properties of equivalent state pairs and unreachable states. 



\chapter{Zusammenfassung}

Die Theorie endlicher deterministischer Automaten (DEAs) ist ein klassisches Thema der Lehre mit Informatikbezug. Eine typisches Problem das Studenten hier lösen sollen ist: Gegeben einen Aufgaben-DEA $A_{task}$, minimiere ihn. Das Generieren solcher Minimierungsprobleme wird allerdings häufig manuell vom Übungsleiter vorgenommen. In dieser Arbeit werden somit Ideen präsentiert um DEA Minimierungsprobleme automatisiert zu generieren.

Wir beginnen in Kapitel~\ref{ch:2} mit einigen einführenden Definitionen, darunter insbesondere Hopcroft's Minimierungsalgorithmus. Auf dessen Basis werden wir Voraussetzungen, Parameter und Struktur eines Programms erarbeiten, das DFA Minimierungsprobleme generiert. Die Struktur dieses Programms wird wie folgt sein: Zunächst wird der Lösungs-DEA generiert, dann werden äquivalente Zustandspaare erzeugt und schließlich unerreichbare Zustände hinzugefügt. Diesen Ansatz kann man grob als Umkehrung von Hopcroft's Algorithmus charakterisieren.

Um die Lösungs-DEAs zu generieren (Kapitel~\ref{ch:3}) machen wir Gebrauch von einem simplen Algorithmus, der wiederholt Test-DEAs mittels Randomisierung oder Enumerierung erzeugt und sie immer dann ablehnt, wenn sie den gewünschten Eigenschaften nicht entsprechen. Aus relevante Ergebnissen verwandter Forschungsarbeit werden wir einige Schlussfolgerungen für diese Arbeit ziehen können.

In Kapitel~\ref{ch:4} beschreiben wir, wie Lösungs-DEAs zu Aufgaben-DEAs erweitert werden können. Um das zu erreichen, können wir Zustände und Transitionen recht einfach mithilfe gewisser Regeln hinzufügen. Diese Regeln werden direkt von den Eigenschaften äquivalenter Zustandspaare und unerreichbarer Zustände abgeleitet.
	