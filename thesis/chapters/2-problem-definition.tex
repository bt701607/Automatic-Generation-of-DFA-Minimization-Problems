% !TeX spellcheck = en_US

\chapter{Problem definition and approach} \label{ch:2}

In this chapter we will set foundations, investigate sensible parameters and requirements for a minimization task generator and deduce our general approach to build such a program.

\section{Preliminaries}

We start with defining preliminary theoretical foundations. By $[[n]]$ we will denote the set of integers $\{0,\ldots,n-1\}$.

\subsection{Deterministic Finite Automatons}

A 5-tuple $A = (Q, \Sigma, \delta, s, F)$ with $Q$ being a finite set of \emph{states}, $\Sigma$ a finite set of \emph{alphabet symbols}, $\delta \colon\ Q \times \Sigma \to Q$ a \emph{transition function}, $s \in Q$ a \emph{start state} and $F \subseteq Q$ \emph{final states} is called \emph{deterministic finite automaton} (DFA)~\cite[p. 46]{HMU01}. From now on $\A$ shall denote the set of all DFAs.

We say $\delta(q,\sigma) = p$ is a transition from $q$ to $p$ using symbol $\sigma$. We define the \emph{extended transition function} $\delta^* : Q \times \Sigma^* \to Q$ of a DFA $A = (Q, \Sigma, \delta, s, F)$ as:
\begin{itemize}
	\item $\delta^*(q,\varepsilon) = q$
	\item $\delta^*(q,w\sigma) = \delta(\delta^*(q,w),\sigma)$ for all $q \in Q$, $w \in \Sigma^*$, $\sigma \in \Sigma$
\end{itemize}
Then, the \emph{language} of $A$ is defined as $L(A) = \{\ w\ |\ \delta^*(w) \in F\ \}$~\cite[pp. 49-50. 52]{HMU01}.

Given a state $q \in Q$. With $d^-(q)$ we denote the set of all \emph{ingoing} transitions $\delta(q', \sigma) = q$ of $q$. With $d^+(q)$ we denote the set of all \emph{outgoing} transitions $\delta(q, \sigma) = q'$ of $q$~\cite[pp. 2-3]{CP05}.

\begin{definition}[(Un-)Reachable State]\label{ch:2:unreachable-states}
	We say a state $q$ is \emph{(un-)reachable} in a DFA $A$, iff there is a (no) word $w \in \Sigma^*$ such that $\delta^*(s, w) = q$.
\end{definition}
\noindent If all states of a DFA $A$ are reachable, then we call $A$ \emph{accessible}~\cite[p. 2]{CP05}.

A DFA is called \emph{complete} iff for all states, every symbol of the alphabet is used on an outgoing transition: $\forall q\in Q\colon \forall\sigma\in\Sigma\colon \exists p\in Q\colon \delta(q,\sigma) = p$. Note, that every incomplete DFA can be converted to a complete one by adding a so called \emph{dead state}~\cite[p. 67]{HMU01}. The resulting automaton has the same language. From now on we will only work with complete DFAs.

\begin{figure}[H]
	\begin{subfigure}{.5\textwidth}\centering\resizebox{1.\linewidth}{!}{
		\begin{tikzpicture}[initial text={},scale=1., every node/.style={transform shape}]
		\tikzstyle{every state}=[minimum size=5mm, inner sep=0pt]
		
		\node[initial, state]  (0) at (0, 0) {$0$};
		\node[accepting,state] (1) at (2, 1.5) {$1$};
		\node[state]           (2) at (2,-1.5) {$2$};
		\node[state]           (3) at (4, 0) {$3$};
		\node[state]           (4) at (4,-1.5) {$4$};
		
		\path[->]
		(0) edge 			  node [above left=-0.15cm]  {$a$}   (1)
		(1) edge [loop above] node [above=-0.07cm]		 {$a$}   (1)
		(1) edge 			  node [left=-0.07cm]        {$b$}   (2)
		(0) edge 			  node [below left=-0.15cm]  {$b$}   (2)
		(2) edge [loop below] node [below=-0.07cm]       {$a$}   (2)
		(3) edge 			  node [above right=-0.15cm] {$a,b$} (1)
		(4) edge [loop below] node [below=-0.07cm]	     {$a$}   (4)
		;
		\end{tikzpicture}
	}\end{subfigure}
	\hfill
	\begin{subfigure}{.4\textwidth}
		\begin{align*}
			&A = (Q, \Sigma, \delta, s ,F) \\
			&\\
			&Q = [[5]] = \{\ 0,1,2,3,4\ \} \\
			&\Sigma = \{\ a,b\ \} \\
			&\delta = \{\ ((0,a),1), ((0,b),2), \ldots\ \} \\
			&s = 0 \\
			&F = \{\ 1\ \}
		\end{align*}
	\end{subfigure}
	\caption{An example DFA. The states $3$ and $4$ are unreachable. This DFA is not complete since the transitions $\delta(2,b)$ and $\delta(4,b)$ are not defined.}
	\label{fig:dfa}
\end{figure}

\begin{definition}[Minimal DFA]
	We call a DFA $A$ \emph{minimal}, if there exists no other DFA with the same language having less states.
\end{definition}
\noindent With $\Amin$ we shall denote the set of all minimal DFAs.

\begin{definition}[Equivalent and Distinguishable State Pairs~{\cite[p. 154]{HMU01}}]\label{ch:2:def:eq-dist-pairs}
	A state pair $q_1, q_2 \in Q$ of a DFA $A = (Q, \Sigma, \delta, s, F)$ is called \emph{equivalent}, iff $\sim_A(q_1, q_2)$ is true, where
	\begin{displaymath}
	q_1 \sim_A q_2 \text{ is true}\ \Leftrightarrow_{def}\ \forall z \in \Sigma^* \colon\ (\delta^*(q_1, z) \in F \Leftrightarrow \delta^*(q_2, z) \in F)
	\end{displaymath}
	If $q_0 \not\sim_A q_1$, then $q_0$ and $q_1$ are called a \emph{distinguishable} state pair. It is well-known (see for instance~\cite[p. 160]{HMU01}) that the relation $\sim_A$ is an equivalence relation.
\end{definition}

\subsection{Isomorphy of DFAs}\label{ch:2:sec:isom}

Given two DFAs $A_1 = (Q_1, \Sigma_1, \delta_1, s_1, F_1)$ and $A_2 = (Q_2, \Sigma_2, \delta_2, s_2, F_2)$. We say $A_1$ and $A_2$ are \emph{isomorphic}, iff:
\begin{itemize}
	\item $|Q_1| = |Q_2|$, $\Sigma_1 = \Sigma_2$ and
	\item a bijection $\pi\colon Q_1 \to Q_2$ exists such that:
	
	$\pi(s_1) = s_2$
	
	$\forall q\in Q_1\colon (q\in F_1 \Longleftrightarrow \pi(q)\in F_2)$
	
	$\forall q\in Q_1\colon \forall\sigma\in\Sigma_1\colon \pi(\delta_1(q,\sigma))=\delta_2(\pi(q),\sigma))$
\end{itemize}
In~\cite[p. 45]{Sch01} we can find the following statement:
\begin{theorem}\label{ch:2:thm:uniq-ism}
	Every minimal DFA is unique (has a unique language) except for isomorphy.
\end{theorem}
\noindent We describe a simple isomorphism test for DFAs in appendix~\ref{ch:app:ism-test}.

\subsection{Hopcroft's Minimization Algorithm}

There are three major algorithms for DFA minimization found by Moore, Brzozowski and Hopcroft, respectively~\cite[p. 2]{BBC10}. On the latter an easy algorithm is based, that is presented in the textbook~\cite{HMU01} and will be described here more precisely. We will call Hopcroft's algorithm `the minimization algorithm' from now on. Its general structure is the following:
\vspace{0.2cm}
\begin{algorithmic}[1] \label{ch:2:minalg}
	\Function{\MinAlg}{$A$}
	\State $A' \gets \RemUnr(A, \CompUnr(A))$
	\State \Return $\RemEq(A', \CompDist(A'))$
	\EndFunction
\end{algorithmic}
\vspace{0.2cm}

\noindent It can be seen that \MinAlg\ works in four major steps. In step 1 and 2 unreachable states are found and removed. In step 3 all equivalent state pairs are found. Step 4 removes states in such a way, that no equivalent state pairs are left.
\begin{enumerate}
	\item Compute all unreachable states via breadth-first search.
	
	\vspace{0.2cm}
	\begin{algorithmic}[1]
		\Function{\CompUnr}{$A$}
			\State $U \gets Q \setminus \{s\}$	\Comment{undiscovered states}
			\State $O \gets \{s\}$				\Comment{observed states}
			\State $D \gets \{\}$				\Comment{discovered states}
			\While {$|O| > 0$}
				\State $N \gets \{\ p\ | \ \exists q \in O\colon\ \sigma \in \Sigma \colon\ \delta(q, \sigma) = p\ \land\ p \notin O \cup D\ \}$
				\State $U \gets U \setminus N$
				\State $D \gets D \cup O$
				\State $O \gets N$
			\EndWhile
			\State \Return $U$
		\EndFunction
	\end{algorithmic}

	\item Remove all unreachable states and their transitions.
	
	\vspace{0.2cm}
	\begin{algorithmic}[1]
		\Function{\RemUnr}{$A, U$}
            \State $\delta' \gets \delta \setminus \{\ ((q,\sigma),p)\in\delta\ |\ q\in U\ \lor\ p\in U\ \}$
			\State \Return $(Q \setminus U, \Sigma, \delta', s, F \setminus U)$
		\EndFunction
	\end{algorithmic}

	\item Compute all equivalent state pairs ($\sim_A$). The representation is inspired by Martens and Schwentick~\cite[ch.~4, p.~17]{MS18}. Note that \CompDist\ requires its input automaton to be complete (\cite[p.~13]{BBC10}).
	\vspace{0.2cm}
	\begin{algorithmic}[1]
		\Function{\CompDist}{$A$} \label{ch:2:minmark}
		\State $M \gets \{ (p,q), (q,p)\ |\ p \in F, q \notin F \}$
		\Do
			\State $M' \gets \{ (p,q)\ |\ (p,q) \notin M \land \exists \sigma \in \Sigma \colon (\delta(p,\sigma), \delta(q,\sigma)) \in M \}$
			\State $M \gets M \cup M'$
		\doWhile {$M' \neq \emptyset$}
		\State \Return $Q^2 \setminus M$
		\EndFunction
	\end{algorithmic}

	\item Merge all equivalent state pairs, which are exactly those in $\sim_A$. The representation is inspired by Högberg and Larsson~\cite[p.~10]{HL20}.
	
	\vspace{0.2cm}
	\begin{algorithmic}[1] \label{ch:2:minmerge}
		\Function{\RemEq}{$A$, $\sim_A$}
            \State $Q' \gets \emptyset, \delta' \gets \emptyset, F' \gets \emptyset$
            \For {$q$ \textbf{in} $Q$}
                \State Add $[q]$ to $Q'$ \Comment{$[\cdot]_{\sim_A}$ shall be abbreviated $[\cdot]$}
                \For {$\sigma$ \textbf{in} $\Sigma$}
                    \State $\delta'([q], \sigma) = [\delta(q, \sigma)]$
                \EndFor
                \If {$q \in F$}
                    \State Add $[q]$ to $F'$
                \EndIf
            \EndFor
			\State \Return $(Q', \Sigma, \delta', [s], F')$
		\EndFunction
	\end{algorithmic}
\end{enumerate}
The following theorem states the most important property of \MinAlg.

\begin{theorem}[{\cite[pp. 162-164]{HMU01}}]\label{ch:2:min-alg-correct}
	\MinAlg\ computes a minimal DFA to its input DFA.
\end{theorem}

\noindent When looking at \CompDist, one notes, that it computes distinct subsets of $Q \times Q$ on the way. Indeed, one could write the algorithm in such a way, that these subsets are explicitly computed in form of a function $m\colon\mathbb{N}\to\mathcal{P}(Q\times Q)$:
\vspace{0.2cm}
\begin{algorithmic}[1] \label{ch:2:m-minmark}
	\Function{\mCompDist}{$A$}
	\State $i \gets 0$
	\State $m(0) \gets \{ (p,q), (q,p)\ |\ p \in F, q \notin F \}$
	\Do
		\State $i \gets i + 1$
		\State $m(i) \gets \{ (p,q), (q,p)\ |\ (p,q) \notin \bigcup{m(\cdot)} \land \exists \sigma \in \Sigma \colon (\delta(p,\sigma), \delta(q,\sigma)) \in m(i-1) \}$
	\doWhile {$m(i) \neq \emptyset$}
	\State \Return $\bigcup{m(\cdot)}$
	\EndFunction
\end{algorithmic}
\vspace{0.2cm}
Using this redefinition, we can easier refer to the state pairs marked in a certain iteration. We will use both variants of \CompDist\ in exchange.
\begin{definition}\label{ch:2:def:D(A)}
	We denote the number of iterations done by \CompDist\ on a DFA $A$ as $\mmD(A)$.
\end{definition}
\noindent This number will prove to be a major characteristic of difficulty for a minimization task.

\section{DFA Minimization Problems}

Now that we have introduced all necessary basic definitions, we present an example DFA minimization task and its sample solution, as it could have been given to students in an introductory course on automata theory. In this context further naming conventions will be determined.

\begin{figure}[ht]
	{\raggedright\itshape \underline{Task:} Consider the below shown deterministic finite automaton $A$:}
	\begin{center}\resizebox{.75\linewidth}{!}{
		\begin{tikzpicture}[initial text={},scale=1., every node/.style={transform shape}]
		\tikzstyle{every state}=[minimum size=5mm, inner sep=0pt]
		
		\node[initial, state]  (0) at (0, 0)    {$0$};
		\node[state] 		   (1) at (6, 0)    {$1$};
		\node[state,accepting] (2) at (4,0)     {$2$};
		\node[state]           (3) at (2, -1.5) {$3$};
		\node[state,accepting] (4) at (4,-3)    {$4$};
		\node[state]           (5) at (4,-1.5)  {$5$};
		\node[state]           (6) at (2,0)     {$6$};
		
		\path[->]
		(0) edge node [above=-0.07cm]  {$a$}   (6)
		(0) edge [bend left] node [above=-0.07cm]  {$b$}   (2)
		
		(1) edge node [above=-0.07cm]  {$a$}   (2)
		(1) edge node [below right=-0.15cm]  {$b$}   (4)
		
		(2) edge [bend left] node [above=-0.07cm]  {$a$}   (1)
		(2) edge node [below right=-0.15cm]  {$b$}   (3)
		
		(3) edge node [left=-0.07cm]  {$a$}   (6)
		(3) edge node [above right=-0.15cm]  {$b$}   (4)
		
		(4) edge [bend right] node [below right=-0.15cm]  {$a$}   (1)
		(4) edge [bend left] node [below left=-0.15cm]  {$b$}   (0)
		
		(5) edge node [below right=-0.15cm]  {$a$}   (1)
		(5) edge node [left=-0.07cm]  {$b$}   (4)
		
		(6) edge [bend right] node [below=-0.07cm]  {$a$}   (1)
		(6) edge node [above=-0.07cm]  {$b$}   (2)
		;
		\end{tikzpicture}
	}\end{center}
	\itshape Apply the minimization algorithm and illustrate for each state pair of $A$ during which \CompDist-iteration it was marked. Draw the resulting automaton.
	\caption{An example DFA minimization task.}
	\label{fig:dfa_ex_task}
\end{figure}

\begin{figure}[ht]
	{\raggedright\itshape \underline{Solution:}\newline
		Step 1: Detect and eliminate unreachable states.
		\begin{tabbing}
			\qquad\textnormal{State $5$ is unreachable.}
		\end{tabbing}
		Step 2: Apply \CompDist\ to $A$ and merge equivalent state pairs:\par
	}
	\begin{subfigure}{.4\textwidth}
		\vspace{0.5cm}
		\qquad
		\begin{tabular}{c|c|c|c|c|c|c}
			  & 0  & 1  & 2  & 3  & 4  & 6  \\\hline
			0 & \x & 1  & 0  &    & 0  & 2  \\\hline
			1 & \x & \x & 0  & 1  & 0  & 1  \\\hline
			2 & \x & \x & \x & 0  &    & 0  \\\hline
			3 & \x & \x & \x & \x & 0  & 2  \\\hline
			4 & \x & \x & \x & \x & \x & 0  \\\hline
			6 & \x & \x & \x & \x & \x & \x \\
		\end{tabular}
	\end{subfigure}
	\begin{subfigure}{.5\textwidth}\centering\resizebox{1.\linewidth}{!}{
		\begin{tikzpicture}[initial text={},scale=1., every node/.style={transform shape}]
		\tikzstyle{every state}=[minimum size=5mm, inner sep=0pt]
		
		\node[initial, state]  (03) at (0, 0)   {$03$};
		\node[state] 		   (1) at (6, 0)   {$1$};
		\node[state,accepting] (24) at (4,0)    {$24$};
		\node[state]           (6) at (2,0)    {$6$};
		
		\path[->]
		(03) edge node [above=-0.07cm]  {$a$}   (6)
		(03) edge [bend left] node [above=-0.07cm]  {$b$}   (24)
		
		(1) edge node [above=-0.13cm]  {$a,b$}   (24)
		
		(24) edge [bend left=50] node [above=-0.07cm]  {$a$}   (1)
		(24) edge [bend right=50] node [above=-0.07cm]  {$b$}   (03)
		
		(6) edge [bend right] node [below=-0.07cm]  {$a$}   (1)
		(6) edge node [above=-0.07cm]  {$b$}   (24)
		;
		\end{tikzpicture}
	}\end{subfigure}
	\caption{Solution to the DFA minimization task in fig.~\ref{fig:dfa_ex_task}.}
	\label{fig:dfa_ex_sol}
\end{figure}

\noindent Figures~\ref{fig:dfa_ex_task} and~\ref{fig:dfa_ex_sol} show a DFA minimization task and its solution. The students are confronted with a \emph{task DFA} $A_{task}$, which has to be minimized --- we will assume that Hopcroft's algorithm is used.

As a consequence (following alg.~\ref{ch:2:minalg}), at first, unreachable states have to be eliminated, so that we gain $A_{re}$ (having only \emph{reachable} states). Secondly, equivalent state pairs of $A_{re}$ are merged so that the minimal \emph{solution DFA} $A_{sol}$ is found. The table $T$ displayed in Figure~\ref{fig:dfa_ex_sol} is nothing else but a visualization of the function $m$ of \mCompDist, where $T(q_0, q_1) = i \Leftrightarrow (q_0, q_1) \in m(i)$.

Notice that step 2 and in particular the computation of $\sim_A$ can be seen as central: Step 1 is relatively easy. Unreachable states can normally be detected visually without running a search algorithm. Moreover algorithms like breadth-first search are often introduced earlier in computer science education. In step 2 computing $\sim_A$ is conceptually harder to grasp than collapsing $A_{re}$ towards the solution DFA.

Let us now formally define DFA minimization problems.
\begin{definition}[DFAMinimization] $ $ \\
	$ $ \vspace{-0.4cm} \\
	\noindent $\underline{\emph{Given:}}$ \emph{A DFA $A_{task}$.}
	
	\vspace{0.2cm}
	\noindent $\underline{\emph{Task:}}$ \emph{Compute $A_{sol} = \MinAlg(A_{task})$.}
\end{definition}
\noindent Our goal in this thesis is to automatically generate instances of this problem. Before starting to think about concrete generation algorithms, we will devise some parameters and requirements to influence the properties of the generated problem.

\section{Requirements and Parameters for a Generation Algorithm}\label{ch:2:requirements-analysis}

We already know of some simple requirements in advance. Firstly, we can state that $A_{sol}$ has to be minimal regarding $A_{re}$ and $A_{task}$. Secondly, the language of $A_{task}, A_{re}$ and $A_{sol}$ must be the same. Lastly, we know that every state of $A_{re}$ needs to be reachable.

% Difficulty adjustment possibilities

Besides some other sensible requirements, we will devise especially parameters for generation algorithms in this section. These parameters are going to be adjustment possibilities, which will in almost all cases allow the user to influence the difficulty of the generated problem in some way.

Since the difficulty of a DFA minimization task $A_{task}$ for students is mainly determined by the difficulty of executing \MinAlg$(A_{task})$, most parameters will influence certain aspects of the execution of \MinAlg.

\paragraph*{\CompDist-iterations ($d$).}

Consider the computation of the sets $m(i)$ in \mCompDist\ (see alg.~\ref{ch:2:m-minmark}). Determining $m(0)$ is quite straightforward, because it consists simply of tests whether two states are in $F \times Q \setminus F$ (see line 3). Determining $m(1)$ is less easy: The rule for determining all $m(i), i > 0$ is different to that for $m(0)$ and more complicated (see line 6). Determining $m(2), m(3), \ldots$ requires the same rule. It shows nonetheless a students understanding of \CompDist' terminating behavior: The algorithm does not stop after computing $m(1)$, but only when no more distinguishable state pairs were found.

It would therefore be sensible if $\mmD(A_{task})$ could be adjusted by some parameter $d$.

\paragraph*{Number of states \texorpdfstring{($\nSO, \nEQ, \nUN$)}{}.}

To control the number of states in $A_{task}, A_{re}$ and $A_{sol}$, we will introduce three parameters: $\nSO, \nEQ, \nUN \in \mathbb{N}$, where
\begin{itemize}
	\item $\nSO$ is the number of states of the \emph{solution} DFA $A_{sol}$
	\item $\nEQ$ is the number of distinct \emph{equivalent} state pairs of $A_{re}$
	\item $\nUN$ is the number of \emph{unreachable} states of $A_{task}$
\end{itemize}
They can be equivalently described by the following equations:
\begin{align*}
    |Q_{sol}| &= \nSO \\
    |Q_{re}| &= \nSO + \nEQ \\
    |Q_{task}| &= \nSO + \nEQ + \nUN
\end{align*}
It is sensible to have $\nUN > 1, \nEQ > 1$, such that \RemUnr\ and \RemEq\ will not be skipped. To not make the task trivial, $\nSO > 2$ is sensible. An exercise instructor will find it useful to control exactly how big $\nUN$, $\nEQ$ and $\nSO$ are: The higher $\nUN, \nEQ$, the more states have to be eliminated and merged. The higher $\nSO + \nEQ$, the more state pairs have to be checked during \CompDist.

\paragraph*{Alphabet size \texorpdfstring{($\kAL$)}{}.}

The more symbols the alphabet of $A_{task}, A_{re}$ and $A_{sol}$ has (note that \MinAlg\ does not change the alphabet), the more transitions have to be followed when checking whether $(\delta(q,\sigma),\delta(p,\sigma))\in m(i-1)$ is true for each state pair $p,q$ (see line 6 of alg.~\ref{ch:2:m-minmark}). In addition we will see later, that there exists no DFA with $k<2$ having any equivalent state pairs, so $k\ge 2$ is sensible and even required if $\nEQ > 0$.

\paragraph*{Number of final states \texorpdfstring{($\nF$)}{}.}

Most DFAs in teaching have about 1 to 3 final states (see e.g.~\cite[pp. 48-78]{HMU01} or~\cite[pp. 28-48]{Sch01}), so being able to set a number of final states allows concentrating on or deviating from familiar DFAs.

In this work $\nF$ shall determine the number of final states in $A_{sol}$. Consequently, $A_{task}$ may have minimum $\nF$ and maximum $\nF + \nEQ + \nUN$ final states. The latter is the case, if all equivalent states are created from final states and if every unreachable state is created as final state.

\paragraph*{Uniqueness of solution DFA language.}

For an exam, it would be sensible to be able to generate a task where the DFA language is unique, meaning there was no previously generated DFA with the same language.

Note that, if $A_{sol}$ is indeed \emph{new} in that sense, then $A_{task}$ will automatically have a unique language too, since $A_{sol}$ and $A_{task}$ always have the same language.

\paragraph*{Completeness of task DFA \texorpdfstring{($c$)}{}.}

In opposition to \CompDist, \CompUnr\ and \RemUnr\ do not require their input DFA $A_{task}$ to be complete. So we could have unreachable states in $A_{task}$, to which $\delta$ is not defined for all alphabet symbols. It is however sensible to build task DFAs complete too to avoid possible confusion: Such subtleties do not highlight the main ideas of \MinAlg.

Nonetheless we shall introduce a boolean parameter $c$, that determines if there are unreachable states, that make $A_{task}$ incomplete. Thus, an exercise lecturer has the option, to showcase this matter on a DFA and generate exercises accordingly.

\paragraph*{Planar drawing of task DFA \texorpdfstring{($p_{sol}, p_{task}$)}{}.}

A graph $G$ is \emph{planar} if it can be represented by a drawing in the plane such that its edges do not cross. Such a drawing is then called \emph{planar drawing} of $G$. A visual aid for students would be given, if the task DFA were planar and presented as a planar drawing. In this work and its associated implementation libraries and parameters $p_{sol}, p_{task} \in \{0,1\}$ (toggling planarity of $A_{sol}, A_{task}$) will be used to allow the option of planarity, but neither ensuring planarity nor planar drawing will be investigated further theoretically.

\paragraph*{Maximum degree of any state in task DFA.}

The \emph{degree} $deg(q)$ of a state $q \in Q$ in a DFA $A$ is defined as $deg(q) = |d^-(q)| + |d^+(q)|$ (the total number of transitions in which $q$ participates). By capping the maximum degree for all states, the graphical representation of the DFA would be more clear. The inclusion of a maximum degree parameter is here omitted.

%Note that $deg(q) \geq |\Sigma|$ for any complete DFA, since states of complete DFAs have to use all alphabet symbols on outgoing transitions.

\section{Approach and General Algorithm}

We will first build the solution DFA (step 1), and --- based on that --- the task DFA by creating equivalent states and adding unreachable states (step 2). Both steps together will fulfill all criteria chosen above and are covered each in depth in chapter~\ref{ch:3} respectively chapter~\ref{ch:4}.

At the beginning of chapter~\ref{ch:3} and~\ref{ch:4}, we will provide formal problem definitions for both steps, that specify precisely all requirements. Here we shall content ourselves with the definition of the main algorithm:
\vspace{0.2cm}
\begin{algorithmic}[1]
	\Function{GenDFAMinProblem}{$\nSO, \kAL, \nF, d, p_{sol}, p_{task}, \nEQ,\nUN, c$}
	\State $A_{sol} \gets \textsc{GenNewMinDFA}(\nSO, \kAL, \nF, d, p_{sol})$
	\State $A_{task} \gets \textsc{ExtendMinDFA}(A_{sol}, p_{task}, \nEQ, \nUN, c)$
	\State \Return $A_{task}$
	\EndFunction
\end{algorithmic}
\vspace{0.2cm}
\noindent One might notice, that we pass $d$ only to $\textsc{GenNewMinDFA}$, even though we want to ensure $d = \mmD(A_{task})$ too. This is a valid approach due to the fact that creating equivalent states and adding unreachable states does not change the $\mmD$-value, which we are going to prove. 

However, since unreachable states are eliminated before \CompDist\ is applied, we need only to prove, that creating equivalent states does not change the $\mmD$-value, which will be done during the discussion of step 2, more specifically in section~\ref{ch:4:sec-D-proof}.


