% !TeX spellcheck = en_US
\documentclass[a4paper, oneside, 11pt]{report}

\usepackage[a4paper,width=150mm,top=25mm,bottom=25mm]{geometry}

%packages for german language
\usepackage[utf8]{inputenc}
\usepackage[english]{babel}
\usepackage[T1]{fontenc}
\usepackage{lmodern}

\usepackage{graphicx}
\graphicspath{ {images/} }

\usepackage{amsmath}
\usepackage{amsthm}
\usepackage{amssymb}

\usepackage{algorithm}
\usepackage[noend]{algpseudocode} % [noend]

% ------------------------

\newtheorem{theorem}{Theorem}
\newtheorem{corollary}{Corollary}
\newtheorem{lemma}{Lemma}

\theoremstyle{definition}
\newtheorem{definition}{Definition}

\theoremstyle{remark}
\newtheorem*{remark}{Remark}

\algdef{SE}[DOWHILE]{Do}{doWhile}{\algorithmicdo}[1]{\algorithmicwhile\ #1}%

% Title Page
\title{Generation of DFA Minimization Problems}
\author{Gregor H. C. Sönnichsen}

\begin{document}
\maketitle

%\begin{abstract}
%	Abstract goes here
%\end{abstract}

%\chapter*{Acknowledgements}
%I want to thank...

\tableofcontents

\chapter{Introduction}

\section{Preliminaries}

\begin{itemize}
	\item sink-state, unreachable, lonely, ingoing/outgoing transitions
	\item def of isomorphic dfas
	\item def of Myhill-Nerode-equivalence-relation to a language
	\item def of minimal dfa to a language
	\item statement that minimal dfa is unique besides isomorphism
	\item minimization depth
\end{itemize}

\subsection{Finite Automatons}

\begin{definition}[Deterministic Finite Automaton]
	A 5-tuple $A = (Q, \Sigma, \delta, s, F)$ with $Q$ as set of \emph{states}, $\Sigma$ as \emph{input alphabet}, $\delta \colon\ Q \times \Sigma \to Q$ as \emph{transition function}, $s \in Q$ as \emph{start state} and $F \subseteq Q$ as \emph{finite states} is called \emph{deterministic finite automaton}.
\end{definition}
\noindent From now on $\mathcal{A}$ shall denote the set of all deterministic finite automatons.

\begin{definition}[Nondeterministic Finite Automaton]
	A 5-tuple $A = (Q, \Sigma, \delta, s, F)$ with $Q$ as set of \emph{states}, $\Sigma$ as \emph{input alphabet}, $\delta \colon\ Q \times \Sigma \to \mathcal{P}(Q)$ as \emph{transition function}, $s \in Q$ as \emph{start state} and $F \subseteq Q$ as \emph{finite states} is called \emph{nondeterministic finite automaton}.
\end{definition}

\subsection{Minimization algorithm}

\begin{algorithmic}[1]
	\Function{MinimizationMark}{$(Q, \Sigma, \delta, s, F)$}
	\State $M \gets \{ (p,q), (q,p)\ |\ p \in F, q \notin F \}$
	\Do
		\State $M' \gets \{ (p,q)\ |\ (p,q) \notin M \land \exists \sigma \in \Sigma \colon (\delta(p,\sigma), \delta(q,\sigma)) \in M \}$
		\State $M \gets M \cup M'$
	\doWhile {$M' \neq \emptyset$}
	\State \Return $M$
	\EndFunction
\end{algorithmic}

\subsection{Duplicate States}

\begin{definition}[Duplicate States]
	Two states $q_1, q_2 \in Q$ of a finite automaton $A = (Q, \Sigma, \delta, s, F)$ are called \emph{duplicates} of each other, iff $d_A(q_1, q_2)$ is true, whereas
	\[
	d_A = \{\ (p, q)\ |\ \forall z \in \Sigma^* \colon\ \delta^*(p, z) \in F \Leftrightarrow \delta^*(q, z) \in F\ \}
	\]
\end{definition}
\noindent Note that \textsc{MinimizationMark} does compute exactly $N_A(p,q) = \neg d_A(p, q)$.

\section{Problem Definition}

heuristic:
\begin{itemize}
	\item $h \colon \mathcal{A} \times \mathcal{A} \to \mathbb{R^+}$
	\item $h(A_{min}, A_{task}) = studentfriendliness$
\end{itemize}

\section{Structure of this Work}



\chapter{DFA requirements and following implications}

\section{Heuristics for the friendlieness of dfas}

\begin{itemize}
	\item $h(A_{min}, A_{task}) = |Q_{task}|\ /\ |Q_{min}|$
	\item number of transitions
	\item max degree of a node
	\item planarity (can be checked in $O(|Q_{task}|)$)
	\item Does GraphViz have a heuristic?
	\item[->] number of duplicate states
	\item[->] number of minimization algorithm iterations ($depth_min(A)$)
	\item[->] number of unreachable states
\end{itemize}

\section{Proofs for validity of separate steps}

\begin{theorem}[]
	Adding duplicate states to an automaton $A$ does not increase the number of iterations in the \textsc{MinimizationMark}-algorithm for $A$.
\end{theorem}

\begin{proof}
	\begin{description}
		\item
		
		Proof per contradiction.
		
		For clarity's sake we will denote the number of iterations $n \in \mathbb{N}$ done by \textsc{MinimizationMark} on an input deterministic finite automaton $A$ as $minmarkDepth(A)$.
		
		Let's assume adding a duplicate state $q_d$ to a given automaton $A = (Q, \Sigma, \delta, s, F)$ results in an automaton $A' = (Q', \Sigma, \delta', s, F')$ whereas $minmarkDepth(A') > minmarkDepth(A)$.
		
		Concerning $A'$ we can say the following:
		\begin{itemize}
			\item $Q' = Q \cup \{ q_d \}$
			\item $\delta \subseteq \delta'$
			\item $F \subseteq F'$
			\item $\exists q \in Q \colon\ d_A(q, q_d)$
		\end{itemize}
		
		\begin{lemma}
			\begin{multline*}
				minmarkDepth(A) =\ n \Leftrightarrow \\
				n = \max_{n \in \mathbb{N}}\ \ \exists p, q \in Q\ \ \exists w \in \Sigma^* \colon \\
				|w| = n - 1 \land (\delta^*(p,w) \in F \Leftrightarrow \delta^*(q,w) \notin F)
			\end{multline*}
		\end{lemma}
	
		\ldots
	\end{description}
\end{proof}

\begin{lemma}[]
	Adding unreachable states to an automaton, does alter the number of iterations in the minimization-marking-algorithm in the context of the minimization algorithm.
\end{lemma}

\begin{proof}
	content \ldots
\end{proof}

As a result, we will set $depth\_min(A)$ when building the minimal DFA.

We will add duplicate and unreachable states afterwards.


\chapter{Building minimal DFAs}

When building the minimal DFA, we do not want to have duplicates.

\begin{align*}
	f(0) &= F \\
	f(n + 1) &= \{\ q\ |\ \exists p\in f(n),\ \sigma\in\Sigma\colon\ \delta(q,\sigma) = p\ \}
\end{align*}
Based on this we define

\chapter{Extending minimal DFAs}

First, a systematic study of how to extend minimal DFAs will be done. Afterwards, we will define an extension algorithm, which will use the previously found results.

\section{Adding new elements to DFAs}

Our study of DFA extension possibilities will focus on methods, that add or remove transitions or states. In general, we could also change start and accepting nodes, but we will exclude these possibilities here. As a consequence, we may now classify our options as follows:
\begin{enumerate}
	\item Add states without transitions
	\item Add transitions without states
	\item Add states and transitions such that
	\begin{itemize}
		\item At least one state is added
		\item No new state has no transitions
	\end{itemize}
	\item Remove/add states and transitions such that
	\begin{itemize}
		\item At least one state or transition is removed
	\end{itemize}
\end{enumerate}
Adding states $p_1, \ldots, p_n$ leads to the situation, that $p_1, \ldots, p_n$ are lonely states. Since adding lonely states does not affect an automatons language, we can use this as option to extend DFAs.

Adding new transitions does not work because of minimal automaton isomorphism \ldots

For option 3, we can not tell yet, whether it will generate usable automatons. The two additional conditions guarantee, that the generated automatons are distinct from the ones generated by options 1 and 2 \\

\noindent One could imagine, that removing transitions/states and adding new ones again might generate distinct automatons in relation to option 3. However, we can prove, that each automaton generated by this technique is isomorphic to an automaton generated by option 3.

\begin{proof}
	To a given language $L$ only one minimal automaton $A_L$ (despite isomorphism) exists.
	So every "remove-add"-automaton $A_{ra}$ can be transformed to $A_L$ using the minimization algorithm.
	The minimization algorithm does in particular delete some states and transitions.
	Thus adding these states and transitions to $A_L$ is the way to simulate the generation via "remove-add" through generation via "just-add".
\end{proof}
\noindent As a consequence, we will discuss from now on only option 3 in more detail.

\section{Add states and transitions}

We classify several subcases:

\begin{enumerate}
	\item New states have ingoing transitions only
	\begin{enumerate}
		\item All new states are non-accepting
		\item At least one state is accepting
	\end{enumerate}
	\item New states have outgoing transitions only
	\item New states have both in- and outgoing transitions
\end{enumerate}
Adding states $p_0, \ldots, p_n$ with ingoing transitions is okay, if every $p_i$ is non-accepting and thus a sink-state.

We prove that adding accepting states with ingoing transitions only leads to a NFA or to a DFA with a different language.
\begin{proof}
	If a new state $p$ is accepting and if w.l.o.g.\ $\delta(q, \sigma) = p$, then there are several cases:
	\begin{itemize}
		\item $p$ is unreachable: Then we got at least one unreachable sub-graph consisting of more than one states
		\item $p$ is reachable: Then the language of the original automaton (call it $A_{orig}$) is not preserved. This is because of the following:
		
		If $A_{orig}$ were in state $q$ after reading $w \in \Sigma^*$ with $\sigma$ as next Symbol, then there would have to exist a transition $\delta(q, \sigma) = p'$ whereas $p' \in F$ is an accepting state. But this would imply that $A_{orig}$ is an NFA (contradiction).
	\end{itemize}
\end{proof}

\noindent 2. leads to the new state being unreachable or the dfa having a different language.

Thus we remain with option 3, which states that every new state has at least one ingoing transition and at least one outgoing transition.

\section{New states have both in- and outgoing transitions}

\section{Controlling number of minimization iterations}

We define a function, that defines the set of states marked for each round of the minimization algorithm. \\
$f_A \colon \mathbb{N} \to \mathcal{P}(Q^2)$ \\
$f_A(0) = \{ (p,q) | p \in F, q \notin F \}$ \\
$f_A(n+1) = \{ (p,q) | (p,q) \notin \bigcup\limits_{0 \dots n}f(n), \exists \sigma \in \Sigma \colon (\delta(p,\sigma), \delta(q,\sigma)) \in \bigcup\limits_{0 \dots n}f(n)\}$ \\
This function can be naturally computed alongside the minimization algorithm.


\section{The DFA extension algorithm}

\subsection{Usable results}

1. Add states only -> unreachable and sink-state \\
2. Add transitions only -> does not increase state number \\
3. Add/Remove states and transitions -> generated by 4. too \\
4.1. Adding states with ingoing transitions only -> sink-states, NFA or different lang. \\
4.2. Adding states with outgoing transitions only -> unreachable \\
4.3.1. Adding states and duplicate at least its ingoing transitions -> NFA \\
4.3.2. Adding states, split ingoing, duplicate outgoing -> generated by 4.3.2.1. too \\
4.3.2.1. Adding states, split ingoing, split outgoing s.t. $[\delta(q', \sigma_i)]_{\equiv_L} = [\delta(q'', \sigma_i)]_{\equiv_L}$ -> Ok


\subsection{The algorithm}




\chapter{Display DFAs}




\chapter{Conclusion}

What happens, if we change start and accepting states? \\
What happens, if we add transitions only?


\appendix
\chapter{List of Figures}
\chapter{Bibliography}

\end{document}          
