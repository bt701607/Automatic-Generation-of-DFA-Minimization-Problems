% !TeX spellcheck = en_US

\chapter{Introduction} \label{ch:1}

% study computer science, (theoretical informatics), automata theory, value of this theory

Automata theory is recommended as part of a standard computer science curriculum~\cite[pp. 5-6]{GI16}. It provides the chance to gain a precise cognitive model of a theory, possibly yielding new perspectives on other problems and topics. This may thus lead to increased problem solving skills and more accurate thinking.

% typical topic - minimization, (why typical)

A typical task in automata theory is the minimization of a given deterministic finite automaton (DFA). The classic textbook ``Introduction to automata theory, languages, and computation'' by Hopcroft et al.~\cite{HMU01} presents a practicable minimization algorithm. We confine ourselves to look at DFA minimizations using that algorithm.

% sketch situation

In an introduction course to theoretical computer science minimization tasks are thus likely to occur in exercises or exams. As of the creation of such tasks, one may assume, that it is done mostly manually. Automation would yield here the following advantages:

\begin{itemize}
	\item freeing time for other things, e.g. research, helping students face-to-face, designing exercise sheets
	
	\item increased predictability and consistency of the generated task properties, which can be adjusted accurately through various parameters
	
	\item saves humans from generating those tasks which involves monotonous work
\end{itemize}
\gregor{Delete or find externally from Wikipedia}
Engagement on this topic promises moreover increased clarification which kind of minimization tasks can be generated, and where difficulties of those tasks lie.

This work aims to provide theoretical foundations for a DFA minimization task generator. What requirements a user has towards such a program will be discussed in a short requirements analysis. Based on this work a DFA minimization generator will be devised. Alongside to this thesis an implementation of such a generator has been developed. It can be found at \url{https://github.com/bt701607/Generation-of-DFA-Minimization-Problems}.
